\Problema{5}{}
\medskip

Un circuito lineal con tensión de entrada $v_i(t)$ y tensión de salida $v_o(t)$ tiene la siguiente función de transferencia:

\begin{eqnarray*}
H(s) = \dfrac{s+4}{s(s+2)}
\end{eqnarray*}

\begin{subpunto}
	\item Detemine la respuesta al impulso del circuito. \partialPoints{3}.

	\shortsolution{
	\bigskip
	\textbf{Respuesta:}
	\begin{equation*}
		h(t)=(2-e^{-2t})u(t)
	\end{equation*} 
	}	
	
	\item Determine la tensión de salida $v_o(t)$ del circuito descrito por la función $H(s)$ si a éste se le aplica la señal de entrada $v_i(t)=\delta(t)-4e^{-4t}u(t)\,$V.\partialPoints{2}.
	
	\shortsolution{
	\bigskip
	\textbf{Respuesta:}
	\begin{equation*}
		v_o(t)=e^{-2t}u(t)\,V
	\end{equation*} 
	}	
	
\end{subpunto}
