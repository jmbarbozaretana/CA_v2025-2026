%% Quiz 4 Tutoría S2 2022
%% José Miguel Barboza Retana


\Problema{12}{Respuesta en Frecuencia}

\medskip

Sea el circuito

\begin{center}
    \includegraphics[scale=1.2]{prob1}
\end{center}

Considerando $v_i(t)$ como la señal de entrada y $v_o(t)$ la señal de salida en el circuito anterior.

\begin{subpunto}
	\item Determine el tipo de comportamiento en frecuencia que presenta el circuito anterior respecto a la frecuencia (paso bajas, pasa altas, pasabanda o rechaza banda). Justifique su respuesta. \partialPoints{2}
	
	\shortsolution{
	\bigskip
	\textbf{Respuesta:}
	Presenta un comportamiento paso bajas. 
	}
	
	\item Determine la función de respuesta en frecuencia $\mathbf{H}(\omega)=\mathbf{V}_o(\omega)/\mathbf{V}_i(\omega)$.\partialPoints{4}
	
	\shortsolution{
	\bigskip
	\textbf{Respuesta:}
	\begin{equation*}
		\mathbf{H}(\omega) = \dfrac{1}{1+j\omega 3RC+(j\omega)^2R^2C^2}
	\end{equation*} 
	}
	
	\item Considerando $R=1\,$k$\Omega$ y $C=10\mu$F, determine el ancho de banda $B$ de la respuesta en frecuencia $\mathbf{H}(\omega)$. \partialPoints{3}
	
	\shortsolution{
	\bigskip
	\textbf{Respuesta:}
	\begin{equation*}
		\omega_0 = 37.4\,rad/s
	\end{equation*} 
	}	
	
	\item Considerando $R=1\,$k$\Omega$ y $C=10\mu$F, trace el diagrama asintótico de Bode tanto de magnitud como de fase de la función de respuesta en frecuencia $\mathbf{H}(\omega)$. \partialPoints{3}
	
	\shortsolution{
	\bigskip
	\textbf{Respuesta:}
	\begin{center}
		\includegraphics[scale=1]{prob1_sol}
	\end{center} 
	}	
	
\end{subpunto}
