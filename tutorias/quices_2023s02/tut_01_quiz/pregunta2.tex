%% Quiz 1 Tutoría S2 2023
%% Laura Cabrera

\Pregunta{6}{Para el siguiente circuito considere que $v_s(t)=3\cos(t)\,V$.}

\begin{center}
  \includegraphics[scale=1]{p02}
\end{center}

Determine la onda de tensión $v_o(t)$ y la potencia promedio consumida por dicha resistencia de $3\,\Omega$. 
    

\shortsolution{

\bigskip
\textbf{Respuesta:}
    \begin{align*}
    v_o(t) &= 3.18\cos(t+45^o)\,V\\
    P &= 1.6875\,W
    \end{align*}
}


\solution{

\bigskip
\textbf{Solución:}

Al pasar al dominio fasorial se obtiene

\begin{center}
	\includegraphics[scale=1]{p02 - resp}
\end{center}


La tensión en $\mathbf{V}_a$ se puede obtener mediante un equivalente:

\begin{align*}
\mathbf{Z}_{eq} &= \dfrac{1}{\frac{1}{3+1}+\frac{1}{4j}}\\
&=2\sqrt{2}\angle 45^o=2+2j\,\Omega
\end{align*}

Usandolo para un divisor:

\begin{align*}
	\mathbf{V}_{a} &= \dfrac{\mathbf{V}_{s}(2+2j)}{-2j+2+2j}\\
	&=3+3j=3\sqrt{2}\angle 45^oV
\end{align*}


Nuevamente, con divisor:

\begin{align*}
	\mathbf{V}_{o} &= \dfrac{\mathbf{V}_{a}(3)}{3+1}\\
	&=\dfrac{3\sqrt{2}\angle 45^o(3)}{4}\\
	&=\frac{9\sqrt{2}}{4}\angle 45^o=3.18\angle 45^\circ{}\,V 
\end{align*}

Esta tensión puede ser encontrada también por otros métodos como LCK o  LTK \calif{3\,ptos independientemente del camino}. 

Por lo tanto, $v_o(t)=3.18\cos(t+45^o)\,V\rightarrow $ \calif{1\,pto}.\\

Finalmente, para el cálculo da la potencia promedio, es necesario encontrar la corriente en la resistencia:

\begin{align*}
	\mathbf{I}_{R} &= \dfrac{\mathbf{V}_{o}}{R}=\frac{3\sqrt{2}}{4}\angle 45^o\,A
\end{align*}

Así:

\begin{align*}
	P &=\dfrac{1}{2}V_mI_m\cos(\theta_v-\theta_i)\\
	  &=\left(\dfrac{1}{2}\right)\left(\dfrac{9\sqrt{2}}{4}\right)\left(\dfrac{3\sqrt{2}}{4}\right)\cos(0)\\
	  &=1.6875\,W \rightarrow\calif{2\,ptos}
\end{align*}
}
