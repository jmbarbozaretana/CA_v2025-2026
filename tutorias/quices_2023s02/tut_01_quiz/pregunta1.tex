%% Quiz 1 Tutoría S22023
%% Laura Cabrera Quiros

\Pregunta{4}{Sea el siguiente circuito con una fuente de corriente alterna que trabaja a una frecuencia angular de $2\,rad/s$.}

\begin{center}
	\includegraphics[scale=1]{p01}
\end{center}


Se sabe que la onda de tensión en la resistencia $v_R(t)$ tiene una amplitud igual a $0.24\,V$ y está adelantada $45^o$ con respecto a la referencia estándar. Considerando además que el valor de la resistencia es de $100\,\Omega$ y el capacitor de $470\,\mu F$, determine la onda de tensión $v_c(t)$ para el capacitor.\\

\shortsolution{

\bigskip
\textbf{Respuesta:}
    \begin{align*}
    v_c(t)= 2.55\cos(2 t -45^{\circ})\,V
    \end{align*}
}

\solution{
\textbf{Solución:}

Lo primero es encontrar el fasor de $\mathbf{V}_R$, que según la información sería:

\begin{equation*}
	\mathbf{V}_R=0,24\angle 45^{\circ}\,V \rightarrow \calif{1\,pto}
\end{equation*}
	
El fasor de la corriente se puede encontrar a partir de ley de Ohm en la resistencia, y sería:

\begin{equation*}
	\mathbf{I}=\dfrac{0,24\angle 45^o}{100}=2,4\angle{45^{\circ}}\,mA\rightarrow \calif{1\,pto}
\end{equation*}

Al ser un circuito en serie, esta es la misma corriente que pasa por el capacitor. Por ende:

\begin{align*}
	\mathbf{V}_C &= \dfrac{\mathbf{I}}{j\omega C}\\
	\mathbf{V}_C &= \dfrac{2,4x10^{-3}\angle{45^{\circ}}}{j(2)(470\times 10^{-6})}\\
	\mathbf{V}_C &= 2.55\angle{-45^{\circ}}\,V\rightarrow \calif{1.5\,ptos}
\end{align*}

Por lo tanto $v_c(t) = 2.55\cos(2t -45^{\circ})\,V\rightarrow \calif{0.5\,ptos}$. 

}
