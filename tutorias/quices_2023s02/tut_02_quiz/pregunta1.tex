%% Quiz 2 Tutoría S22023
%% Saúl Guadamuz Brenes

\Pregunta{5}{Una fuente balanceada en $Y$ se conecta a una carga balanceada en $\Delta$. Se sabe que la corriente de línea tiene una magnitud de $5 \, A_{rms}$. Si la potencia total consumida por la carga es de $3.12 \, kW$ y el ángulo de cada una de las impedancias de carga es de $-30^\circ$, calcule la magnitud de la tensión de línea y el valor de las impedancias de carga.}

\shortsolution{

\bigskip
\textbf{Respuesta:}
    \begin{align*}
    	V_L &= 416 \, V_{rms} \\
    	\phr{Z_{\Delta}} &= 144.19 \angle -30^\circ \, \Omega = 124.872 - j72.092 \, \Omega
    \end{align*}
}

\solution{
\textbf{Solución:}

Sabemos que la potencia total es $P_T = \sqrt{3}V_L I_L \cos \theta$, por lo que la tensión de línea es
\begin{align*}
	V_L &= \dfrac{P_T}{\sqrt{3} I_L \cos \theta} = \dfrac{3120}{\sqrt{3}(5)\cos (-30^\circ)} \\
			&= 416 \, V_{rms} \quad \calif{1\,pts}
\end{align*}

En una conexión $Y-\Delta$ sabemos que en la carga $I_L = \sqrt{3}I_p$, entonces
\begin{align*}
	I_p &= \dfrac{I_L}{\sqrt{3}} = \dfrac{5}{\sqrt{3}} \\
			&= 2.886 \, A_{rms} \quad \calif{2\,pts}
\end{align*}

La magnitud de la impedancia sería
\begin{equation*}
	Z_\Delta = \dfrac{V_L}{I_L} = \dfrac{416.133}{2.886} = 144.19 \, \Omega \quad \calif{1\,pts}
\end{equation*}
y el enunciado brinda el ángulo, así es que las impedancias de carga son
\begin{equation*}
	\phr{Z}_\Delta = 144.19 \angle -30^\circ \, \Omega = 124.872 - j72.092 \, \Omega \quad \calif{1\,pts}
\end{equation*}

Se hace meritorio mencionar que esta no es la única forma de resolver el ejercicio.

\clearpage
}
