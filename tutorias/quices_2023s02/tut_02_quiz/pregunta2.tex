%% Quiz 1 Tutoría S2 2023
%% Saúl Guadamuz Brenes

\Pregunta{5}{Determine la función de transferencia a partir del diagrama de Bode que se muestra a continuación:}

\begin{center}
  \includegraphics[scale=2]{p02}
\end{center}

Brinde su respuesta en la forma
\begin{equation*}
	\phr{H}(s) = \dfrac{K(s + z_1)(s + z_2) \cdots}{(s + p_1)(s + p_2) \cdots} \bigg\vert_{s=j\omega}
\end{equation*}

Considere que el factor de ganancia constante es mayor a 0.
    
\shortsolution{

\bigskip
\textbf{Respuesta:}

\begin{equation*}
	\phr{H}(s) = \dfrac{100(s + 2)(s + 500)}{(s + 20)(s + 5000)} \bigg\vert_{s=j\omega}
\end{equation*}
}


\solution{
\bigskip
\textbf{Solución:}

Del diagrama de Bode se identifican:
\begin{itemize}
	\item Ceros:
		\begin{itemize}
			\item $1$ simple $@\, \omega = 2 \, rad/s$.
			\item $1$ simple $@\, \omega = 500 \, rad/s$.
		\end{itemize}
	\item Polos:
		\begin{itemize}
			\item $1$ simple $@\, \omega = 20 \, rad/s$.
			\item $1$ simple $@\, \omega = 5000 \, rad/s$.
		\end{itemize}
\end{itemize}

Por lo que la función de transferencia es
\begin{align*}
	\phr{H}(\omega) &= \dfrac{(1 + j\omega/2)(1 + j\omega/500)}{(1 + j\omega/20)(1 + j\omega/5000)} \\
								&= \dfrac{20\cdot 5000(2 + j\omega)(500 + j\omega)}{2\cdot 500(20 + j\omega)(5000 + j\omega)} \\
								&= \dfrac{100 (j\omega + 2)(j\omega + 500)}{(j\omega + 20)(j\omega + 5000)}
\end{align*}
que en la forma solicitada sería
\begin{equation*}
	\phr{H}(s) = \dfrac{100 (s + 2)(s + 500)}{(s + 20)(s + 5000)} \bigg\vert_{s=j\omega}
\end{equation*}

\calif{1\,pto por cada factor polo y cero representado y 1\,pto por la correcta representación final como $H(s)$}

}
