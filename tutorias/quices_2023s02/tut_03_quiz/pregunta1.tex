%% Quiz 3 Tutoría S22023
%% L. Cabrera/William Quirós

\Pregunta{5}{Sea $v(t)$ una señal de tensión periódica con periodo igual a $2\pi$ segundos, la cual se puede representar por medio de una serie de Fourier de la siguiente manera:}

\begin{equation*}
	v(t) = 1+\sum_{k=1}^{4}\left[\dfrac{n}{2}\cos(nt)-\dfrac{2n}{\pi}\sen(nt)\right]\,V
\end{equation*}

con $n=2k$. Recordando que tanto $n$ como $k$ son números enteros.

\begin{enumerate}

\item Determine la frecuencia fundamental y el valor CD de $v(t)$. \partialPoints{1}
	
\shortsolution{
\textbf{Respuesta:}
\begin{align*}
	V_{CD} &= 1 \, V \\
	\omega_0 &= 1 \,rad/s
\end{align*}
}
	
\solution{	
\textbf{Solución:}
El coeficiente $a_0$ ($n=0$) o valor CD de la señal como se observa en la serie dada es $1\,V$.
	
Por su parte, la frecuencia de las bases está dada por $\omega_n=\omega_0 n$. Puede verse que el valor $\omega_0$ que acompaña al $n$ es $1\,rad/s$, con la única particularidad que solo se presentan valores de coeficientes diferentes de cero para $n$ par, es decir $n=2k$.
		
}
	

\item Grafique el espectro para la  magnitud $A_n$ y la fase $\phi_n$ de $v(t)$ hasta el cuarto armónico. \partialPoints{4}


\solution{
\bigskip
\textbf{Solución:}

Para encontrar los espectros, primero se debe transformar la serie dada a la representación de amplitud y fase.
	
\begin{align*}
	A_n\angle\phi_n &= a_n-jb_n\\
		            &= \dfrac{n}{2}+j\dfrac{2n}{\pi}\\
					&= \sqrt{\left(\dfrac{n}{2}\right)^2+\left(\dfrac{2n}{\pi}\right)^2}\angle\tan^{-1}\left(\dfrac{\dfrac{2n}{\pi}}{\dfrac{n}{2}}\right)\\
					&= \dfrac{n\sqrt{\pi^2+16}}{2\pi}\angle\tan^{-1}\left(\dfrac{4}{\pi}\right)\\
\end{align*}
	
	
Por lo tanto, $A_n=\dfrac{n\sqrt{\pi^2+16}}{2\pi}$ y $\phi_n=\tan^{-1}\left(\dfrac{4}{\pi}\right)$ \textbf{solamente} para valores pares de $n$ y  0 para los impares. 

Así, los espectros se ven de la siguiente manera:

\begin{center}
	\includegraphics[scale=1.5]{p01_sol_An}
	\vspace{1.5cm}
	\includegraphics[scale=1.5]{p01_sol_phin}
\end{center}

\calif{2\,ptos por cada espectro}

}

\shortsolution{
\textbf{Respuesta:}

\begin{center}
\includegraphics[scale=1.5]{p01_sol_An}
\vspace{1.5cm}
\includegraphics[scale=1.5]{p01_sol_phin}
\end{center}
}

\end{enumerate}

