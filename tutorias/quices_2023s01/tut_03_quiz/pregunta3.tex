%% Quiz 3 Tutoría S12023
%% Saúl Guadamuz Brenes

\Pregunta{7}{Para la siguiente respuesta en frecuencia}

\begin{equation*}
	\mathbf{H}(\omega) = \dfrac{240 \sqrt{10}(5 + j\omega)}{(12 + j\omega)(25 + j\omega)(40 + j\omega)},
\end{equation*}

Determine:
\begin{itemize}
\item Las expresiones de magnitud $H(\omega)_{dB}$ y fase $\theta(\omega)$.
\item Dibuje el diagrama asintótico de Bode en magnitud y fase.
\end{itemize} 


\solution{
\textbf{Solución:}

Reescribiendo la respuesta en frecuencia a su forma canónica:
\begin{align*}
	\mathbf{H}(\omega) &= \dfrac{240 \sqrt{10} (5)(1 + j\omega / 5)}{(12)(25)(40) (1 + j\omega /12)(1 + j\omega /25)(1 + j\omega /40)} \\
	\mathbf{H}(\omega) &= \dfrac{(\sqrt{10} / 10) (1 + j\omega / 5)}{(1 + j\omega /12)(1 + j\omega /25)(1 + j\omega /40)}
\end{align*}
\calif{1\,pto por forma estándar}

\bigskip
\underline{Respuesta en magnitud:} 

\begin{align*}
H(\omega)_{dB} &= 20\log_{10}\left(\sqrt{10} / 10\right) + 20\log_{10}\left(|1 + j\omega / 5|\right) - 20\log_{10}\left(|1 + j\omega / 12|\right) - 20\log_{10}\left(|1 + j\omega / 25|\right) - 20\log_{10}\left(|1 + j\omega / 40|\right) \\
H(\omega)_{dB} &= -10 + 20\log_{10}[\sqrt{1 + (\omega / 5)^2}]- 20\log_{10}[ \sqrt{1 + (\omega / 12)^2} ] - 20\log_{10}[ \sqrt{ 1 + (\omega / 25)^2} ] - 20\log_{10}[ \sqrt{1 + (\omega / 40)^2} ]
\end{align*}
\calif{1\,pto por cualquiera de las dos expresiones de magnitud anteriores}

\bigskip
\underline{Respuesta en fase:}

\begin{align*}
\theta(\omega) &= 0^{\circ} + \arctan \left( \dfrac{\omega}{5} \right)- \arctan \left( \dfrac{\omega}{12} \right) - \arctan \left( \dfrac{\omega}{25} \right) - \arctan \left( \dfrac{\omega}{40} \right)
\end{align*}
\calif{1\,pto por la expresión de fase anterior}

\bigskip

Por lo tanto, las asíntotas de magnitud son:
\begin{itemize}
	\item Línea recta en $-10 \, dB$.
	\item Línea a $+20 \, dB / dec$ para $\omega > 5$.
	\item Línea a $-20 \, dB / dec$ para $\omega > 12$.
	\item Línea a $-20 \, dB / dec$ para $\omega > 25$.
	\item Línea a $-20 \, dB / dec$ para $\omega > 40$.
\end{itemize}

Mientras que las asíntotas de fase son:
\begin{itemize}
	\item Línea en $0^\circ$.
	\item Por el cero en $5$:
		\begin{equation*}
			\left\{ \begin{array}{rl}
 				0^\circ & \omega < 0.5 \\
  				45^\circ & \omega = 5 \\
  				90^\circ & \omega > 50
       		\end{array} \right.
		\end{equation*}
	\item Por el polo en $12$:
		\begin{equation*}
			\left\{ \begin{array}{rl}
 				0^\circ & \omega < 1.2 \\
  				-45^\circ & \omega = 12 \\
  				-90^\circ & \omega > 120
       		\end{array} \right.
		\end{equation*}
	\item Por el polo en $25$:
		\begin{equation*}
			\left\{ \begin{array}{rl}
 				0^\circ & \omega < 2.5 \\
  				-45^\circ & \omega = 25 \\
  				-90^\circ & \omega > 250
       		\end{array} \right.
		\end{equation*}
	\item Por el polo en $40$:
		\begin{equation*}
			\left\{ \begin{array}{rl}
 				0^\circ & \omega < 4 \\
  				-45^\circ & \omega = 40 \\
  				-90^\circ & \omega > 400
       		\end{array} \right.
		\end{equation*}
\end{itemize}

\begin{center}
\includegraphics[scale=1]{bode_p3}
\end{center}

\calif{2\,pts por la gráfica de magnitud y 2\,pts por la de fase. Cada gráfico se puede dividir su calificación en 0, 0.5, 1, 1.5 o 2\,pts según lo logrado de forma correcta por el estudiante}

}

\shortsolution{

\textbf{Respuesta:}

\begin{itemize}

\item $H(\omega)_{dB} = -10 + 20\log_{10}[\sqrt{1 + (\omega / 5)^2}] - 20\log_{10}[ \sqrt{1 + (\omega / 12)^2} ] - 20\log_{10}[ \sqrt{ 1 + (\omega / 25)^2} ] - 20 \log_{10}[ \sqrt{1 + (\omega / 40)^2} ]$

\item $\theta(\omega) = 0^{\circ} + \arctan \left( \dfrac{\omega}{5} \right)
							   - \arctan \left( \dfrac{\omega}{12} \right) - \arctan \left( \dfrac{\omega}{25} \right) - \arctan \left( \dfrac{\omega}{40} \right)$

\end{itemize}

\begin{center}
\includegraphics[scale=1]{bode_p3}
\end{center}


}
