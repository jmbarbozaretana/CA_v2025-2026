%% Quiz 3 Tutoría S12023
%% Saúl Guadamuz Brenes

\Pregunta{1.5}{Complete la siguiente tabla con los valores en decibeles ${G_v}_ {dB}$ correspondientes a los valores dados de ganancias de tensión $G_v$:}

\shortsolution{
\textbf{Respuesta:}
}

\solution{
\textbf{Solución:}

Aplicando $20 \log_{10}(G_v)$ a cada una de las ganancias de tensión se obtiene:
}

\begin{center}
\begin{tabular}{|c|c|}
\hline
\textbf{Ganancia de tensión ($G_v$)} & \textbf{Ganancia de tensión en dB (${G_v}_ {dB}$)} \\[7pt]
\hline
10	& \solution{$20 \log_{10}(10) = 20$}\shortsolution{$20 \log_{10}(10) = 20$}	\\[5pt]
\hline
40	& \solution{$20 \log_{10}(40) = 32.041$}\shortsolution{$20 \log_{10}(40) = 32.041$}\\[5pt]
\hline
250	&	\solution{$20 \log_{10}(250) = 47.958$}\shortsolution{$20 \log_{10}(250) = 47.958$}\\[5pt]
\hline
1	&	\solution{$20 \log_{10}(1) = 0$}\shortsolution{$20 \log_{10}(1) = 0$}\\[5pt]
\hline
0.5	&	\solution{$20 \log_{10}(0.5) = -6.020$}\shortsolution{$20 \log_{10}(0.5) = -6.020$}\\[5pt]
\hline
0.004	&	\solution{$20 \log_{10}(0.004) = -47.958$}\shortsolution{$20 \log_{10}(0.004) = -47.958$}\\[5pt]
\hline
\end{tabular}
\end{center}

\solution{\calif{0.25pts cada valor}}
