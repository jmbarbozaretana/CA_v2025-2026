%% Quiz 1 Tutoría S12023
%% William Quirós Solano

\Pregunta{2}{Dada la suma fasorial de corrientes}

\begin{align*}
\mathbf{I}_3=\mathbf{I}_1+\mathbf{I}_2
\end{align*}
Considerando que $\mathbf{I}_{3}=1+j$ A e $\mathbf{I}_{2}=2j^2 e^{-j90^{\circ}}$ A, determine la onda de corriente alterna $i_{1}(\omega t)$ asociada al fasor $\mathbf{I}_1$.

\shortsolution{

\bigskip
\textbf{Respuesta:}
    \begin{align*}
    i_{1}(\omega t)= \sqrt{2}\cos(\omega t -45^{\circ})\,A
    \end{align*}

}

\solution{
\textbf{Solución:}

\begin{align*}
\mathbf{I}_3 &= \mathbf{I}_1 + \mathbf{I}_2\\
1+j &= \mathbf{I}_1 + 2 \cdot(-1) \cdot e^{-j\frac{pi}{2}}\\
1+j &= \mathbf{I}_1 + -2(-j)\\
1+j-2j &= \mathbf{I}_1\\
\mathbf{I}_1 &= 1-j = \sqrt{2}\angle{-45^{\circ}}
\end{align*}

Por lo tanto $i_1(\omega t) = \sqrt{2}\cos(\omega t -45^{\circ})\,A$

\calif{1\,pto por manejar el fasor $\mathbf{I}_2$ de forma correcta, 0.5\,ptos por el valor correcto de $\mathbf{I}_1$ y 0.5\,ptos por la expresión correcta de $i_1(\omega t)$}
}
