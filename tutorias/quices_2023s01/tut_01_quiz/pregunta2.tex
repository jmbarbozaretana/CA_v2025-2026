%% Quiz 1 Tutoría S1 2023
%% William Quirós Solano

\Pregunta{3}{Sea el siguiente circuito RLC serie y el diagrama fasorial adjunto de la corriente $\mathbf{I}_s$ entregada por la fuente.}

\begin{center}
  \includegraphics[scale=1]{p01}
\end{center}

Considerando $R\neq 0$, $C\neq 0$ y $L\neq 0$, determine las fases ($\theta_R$, $\theta_L$, $\theta_C$) de los fasores respectivos $\mathbf{V}_{R}$, $\mathbf{V}_{L}$ y $\mathbf{V}_{C}$. 
    

\shortsolution{

\bigskip
\textbf{Respuesta:}
    \begin{align*}
    \theta_R &= 132^{\circ}\\
    \theta_L &= 222^{\circ}=-138^{\circ}\\
    \theta_C &= 42^{\circ}
    \end{align*}
}


\solution{

\bigskip
\textbf{Solución:}

El fasor de $\mathbf{I}_s$ se debe expresar como

\begin{align*}
\mathbf{I}_s &= |\mathbf{I}_s|\angle{(180^{\circ}-48^{\circ})}\\
\mathbf{I}_s &= |\mathbf{I}_s|\angle{132^{\circ}}
\end{align*}

\begin{itemize}
\item 
Cómo la tensión $\mathbf{V}_R$ debe estar en fase con $\mathbf{I}_s$, se debe cumplir que  $\theta_R = 132^{\circ}$. \calif{1\,pto}

\item 
En el caso del inductor, la ley de impedancias de Ohm define que $\mathbf{V}_L=j\omega L \mathbf{I}_s$, lo que implica que la tensión del mismo tendrá $90^{\circ}$ de adelante respecto a $\mathbf{I}_s$. Por lo tanto, $\theta_L=132^{\circ}+90^{\circ}=222^{\circ}=-138^{\circ}$.\calif{1\,pto} 

\item 
Para el capacitor, se sabe que en este la tensión sobre el mismo tendrá $90^{\circ}$ de retraso respecto a la corriente $\mathbf{I}_s$, por lo que $\theta_C=132^{\circ}-90^{\circ}=42^{\circ}$. \calif{1\,pto}
\end{itemize}
}