%% Quiz 4 Tutoría S12023
%% Laura Cabrera

\Pregunta{7}{Para el siguiente circuito}

\begin{center}
	\includegraphics[scale=1]{circ}
\end{center}

\begin{subpregunta}
\item Determine la función de transferencia $\mathbf{H}(\omega)=\mathbf{V}_o(\omega)/\mathbf{V}_s(\omega)$ \partialPoints{2}	
	
\solution{
\textbf{Solución:}
Tomando el circuito en el dominio de la frecuencia se tiene:

\begin{center}
\includegraphics[scale=1]{circ_sol}
\end{center}

\begin{align*}
\mathbf{V}_o &= \mathbf{V}_s\cdot \dfrac{j\omega L}{R+j\omega L}\\
\dfrac{\mathbf{V}_o}{\mathbf{V}_s} &= \dfrac{j\omega L}{R+j\omega L}\\
\mathbf{H}(\omega) &= \dfrac{j\omega L}{R+j\omega L}
\end{align*}

\calif{2\,pts}
}	
	
\shortsolution{
\textbf{Respuesta:}

\begin{equation*}
\mathbf{H}(\omega) = \dfrac{j\omega L}{R+j\omega L}
\end{equation*}	
	}

\item Identifique el tipo de filtrado que presenta el circuito. Justifique su respuesta en base a la función de transferencia. \partialPoints{1}	

\solution{
\textbf{Solución:}
A partir de la función de $\mathbf{H}(\omega)$ se tiene que

\begin{align*}
\mathbf{H}(0) &= 0\\
\lim_{\omega \rightarrow \infty} \mathbf{H}(\omega) &= \lim_{\omega \rightarrow \infty} \dfrac{j\omega L}{j\omega\left(\dfrac{R}{j\omega}+L\right)}=\lim_{\omega \rightarrow \infty} = \dfrac{L}{\left(\dfrac{R}{j\omega}+L\right)} = 1
\end{align*}

Por lo tanto, el circuito anterior se comporta como un filtro pasa altas.


\calif{0.5\,pts por el análisis en $\omega = 0$}

\calif{0.5\,pts por el análisis en $\omega = \infty$}

}		

\shortsolution{
\textbf{Respuesta:} Pasa altas
}
	
\item Encuentre la expresión para calcular la frecuencia de corte del filtro y calcule el valor de esta (en Hertz) para $R=100\,\Omega$ y $L=2\,H$. \partialPoints{4}

\solution{
\textbf{Solución:}

La función de respuesta de magnitud de $\mathbf{H}(\omega)$ es:

\begin{align*}
|\mathbf{H}(\omega)| = \dfrac{\omega L}{\sqrt{R^2 + \omega^2L^2}}
\end{align*}

\calif{1\,pto}

Para calcular la frecuencia de corte se plantea:

\begin{align*}
|H(\omega_c)| &= \dfrac{1}{\sqrt{2}}\\
\dfrac{\omega_c L}{\sqrt{R^2 + \omega_c^2L^2}} &= \dfrac{1}{\sqrt{2}}\\
\left[\dfrac{\omega_c L}{\sqrt{R^2 + \omega_c^2L^2}}\right]^2 &= \left[\dfrac{1}{\sqrt{2}}\right]^2\\
\dfrac{\omega_c^2 L^2}{R^2 + \omega_c^2L^2} &= \dfrac{1}{2}\\
2\omega_c^2 L^2 &= R^2 + \omega_c^2L^2\\
\omega_c &= \sqrt{\dfrac{R^2}{2L^2-L^2}}\\
\omega_c &= \sqrt{\dfrac{R^2}{L^2}}\\
\omega_c &= \dfrac{R}{L} 
\end{align*}

Con lo cual, utilizando los valores de $R$ y $L$ se tiene que $\omega_c = 50\,rad/s$ o $f_c=7.96\,Hz$.

\calif{3\,pts distribuidos por el análisis realizado}

}
	
\shortsolution{
\textbf{Respuesta:} Se llega a la expresión $\omega_c = R/L$ y para los valores dados se llega a $\omega_c=50\,rad/s$ o $f_c = 7.96\,Hz$.
}
	
\end{subpregunta}




