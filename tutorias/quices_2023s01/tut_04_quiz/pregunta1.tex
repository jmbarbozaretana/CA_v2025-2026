%% Quiz 4 Tutoría S12023
%% Laura Cabrera

\Pregunta{3}{Para un circuito RLC en serie con $R=1\,k\Omega$, $L=1\,mH$ y $C=3\,\mu F$, encuentre la frecuencia de resonancia y las frecuencias de potencia media, todas expresadas en Hertz.}

\shortsolution{
\textbf{Respuesta:}

\begin{itemize}
\item $f_0=2905.8\,Hz$
\item $f_1=53.03\,Hz$
\item $f_2=159.21\,kHz$
\end{itemize}

}

\solution{
\textbf{Solución:}

Un circuito RLC presenta un comportamiento resonante a la frecuencia de:


\begin{equation*}
	f_0=\dfrac{\omega_0}{2\pi}=\dfrac{1}{2\pi\sqrt{LC}}=\dfrac{1}{\sqrt{2\pi(1\times 10^{-3})(3\times 10^{-6})}} = 2905.8\,Hz
\end{equation*}

\calif{1\,pto}

\bigskip

Además, para el cálculo de las frecuencias de media potencia se tiene que:

\begin{align*}
	f_1 &= \dfrac{\omega_1}{2\pi} = \dfrac{1}{2\pi}\left(-\dfrac{R}{2L}+\sqrt{\left(\dfrac{R}{2L}\right)^2 + \dfrac{1}{LC}}\right) = 53\,Hz\\
	f_2 &= \dfrac{\omega_2}{2\pi} = \dfrac{1}{2\pi}\left(\dfrac{R}{2L}+\sqrt{\left(\dfrac{R}{2L}\right)^2 + \dfrac{1}{LC}}\right) = 159.21\,kHz
\end{align*}

\calif{1\,pto cada frecuencia}
}


