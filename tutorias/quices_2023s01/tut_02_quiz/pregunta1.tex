%% Quiz 2 Tutoría S12023
%% José Miguel Barboza Retana

\Pregunta{4}{Considere el siguiente sistema trifásico de 3 hilos el cual es alimentado por una fuente balanceada con secuencia positiva y tensión de fase $\mathbf{V}_{\text{cn}}=100\angle{-230^{\circ{}}}\,V_{rms}$:}

\begin{center}
  \includegraphics[scale=1]{p1}
\end{center}
    
Además, considere que las impedancias correspondientes de la carga son $\mathbf{Z}_a=j10\,\Omega$, $\mathbf{Z}_b=j10\,\Omega$ y $\mathbf{Z}_c=j10\,\Omega$. Según toda la información anterior, determine la potencia compleja total absorbida por la carga trifásica?

\solution{
\textbf{Solución:}

Las impedancias de línea se encuentran en serie con las impedancias de fase, por lo que la impedancia equivalente es

\begin{align*}
\mathbf{Z}_{eq} &= \mathbf{Z}_L + \mathbf{Z}_p\\
\mathbf{Z}_{eq} &= 1 + j + j10\\
\mathbf{Z}_{eq} &= 1+j11\Omega
\end{align*}

La corriente de línea $\mathbf{I}_c$ se puede calcular como

\begin{align*}
\mathbf{I}_c = \dfrac{\mathbf{V}_{cn}}{\mathbf{Z}_{eq}} = \dfrac{100\angle{-230^{\circ}}}{1+j11} = 9.05\angle{45.19^{\circ}}\,A_{rms}\qquad\rightarrow\qquad\hfill\calif{2\,ptos}
\end{align*}

Luego, la potencia compleja en la carga $\mathbf{Z}_c$ se determina como

\begin{align*}
\mathbf{S}_p = |\mathbf{I}_c|^2\mathbf{Z}_p = (9.05)^2\cdot j10 = 819.03j\,VA
\end{align*}

Por último, la potencia compleja total es 

\begin{align*}
\mathbf{S}_T = 3\mathbf{S}_p = 2.46j\,kVA \qquad\rightarrow\qquad\hfill\calif{2\,ptos}
\end{align*}

\clearpage
}


\shortsolution{
\textbf{Respuesta:}

\begin{align*}
\mathbf{S}_T = 3\mathbf{S}_p = 2.46j\,kVA 
\end{align*}

}