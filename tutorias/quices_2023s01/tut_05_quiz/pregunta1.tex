%% Quiz 5 Tutoría S12023
%% William Quiros

\Pregunta{8}{Dada la gráfica de corriente $i(t)$ que se muestra en la siguiente figura:}

\begin{center}
	\includegraphics[scale=1]{p01}
\end{center}

Responda lo siguiente:

\begin{subpregunta}
\item ¿Presenta la señal alguna simetría? Si su respuesta es afirmativa, indique el tipo de simetría y justifique. \partialPoints{1}

\solution{
\textbf{Solución:}

Presenta simetría par ya que $i(t)=i(-t)$.
}

\shortsolution{
\textbf{Respuesta:}

La señal tiene simetría par ya que $i(t)=i(-t)$, como se puede observar en la gráfica. $\rightarrow\calif{1\,pto}$
}



\item Calcule los coeficientes de la serie trigonométrica de Fourier de $i(t)$. \partialPoints{7}

\solution{
\textbf{Solución:}

El valor promedio de $i(t)$ o $a_0$ se puede determinar para una señal par a partir de 

\begin{align*}
	a_{0}= \dfrac{2}{T}\int_{t_0}^{t_0+T/2}i(t)\,dt
\end{align*}

Observando la gráfica, $T=3$ e $i(t)=1-t$
 
\begin{align*}
	a_{0} &= \dfrac{2}{3}\int_{0}^{1}(1-t)\,dt\\
	a_{0} &= \dfrac{2}{3}{(1-1/2)}\\
	a_{0} &= \dfrac{1}{3}\rightarrow\calif{1\,pto}
\end{align*}

De igual modo se puede obtener como el área bajo la curva en un periodo dividido por el periodo.



Por otro lado, dado que se tiene simetría par, entonces
\begin{align*}
	b_n = 0 \rightarrow\calif{1\,pto}
\end{align*}

por lo que resta calcular $a_n$ dado por

\begin{align*}
	a_n=\dfrac{4}{T}\int_{t_0}^{t_0+T/2}i(t)\cos(n\omega_{0}t)\,dt
\end{align*}

con $\omega_{0}=\dfrac{2\pi}{T}=\dfrac{2\pi}{3}$, entonces la integral es

\begin{align*}
	a_n &= \dfrac{4}{3}\int_{t_0}^{t_0+T/2}(1-t)\cos\left(\frac{2\pi n}{3}t\right)\,dt\\
	a_n &= \dfrac{4}{3}\left[\int_{0}^{1}\cos\left(\frac{2\pi n}{3}t\right)\,dt - \int_{0}^{1}t\cos\left(\frac{2\pi n}{3}t\right)\,dt\right]\\
	a_n &= \dfrac{4}{3}\left[\dfrac{3}{2\pi n}\sen\left(\frac{2\pi n}{3}t\right)\Big|_{0}^{1} - \dfrac{9}{4\pi^2 n^2}\left(\cos\left(\frac{2\pi n}{3}t\right)+\dfrac{2\pi n}{3}t\sen\left(\frac{2\pi n}{3}t\right)\right)\Bigr|_{0}^{1}\right]\\
	a_n &= \dfrac{4}{3}\left[\dfrac{3}{2\pi n}\sen\left(\frac{2\pi n}{3}\right)-\dfrac{9}{4\pi^2 n^2}\left(\cos\left(\frac{2\pi n}{3}\right)+\dfrac{2\pi n}{3}\sen\left(\frac{2\pi n}{3}\right)-1\right)\right]\\
	a_n &= \dfrac{4}{3}\left[\dfrac{3}{2\pi n}\sen\left(\frac{2\pi n}{3}\right)-\dfrac{9}{4\pi^2 n^2}\cos\left(\frac{2\pi n}{3}\right)-\dfrac{3}{2\pi n}\sen\left(\frac{2\pi n}{3}\right)+\dfrac{9}{4\pi^2 n^2}\right]\\
	a_n &= \dfrac{4}{3}\cdot\dfrac{9}{4\pi^2 n^2}\left[1-\cos\left(\frac{2\pi n}{3}\right)\right]\\
	a_n &= \dfrac{3}{\pi^2 n^2}\left[1-\cos\left(\frac{2\pi n}{3}\right)\right]
\end{align*}
\calif{5\,pts que se pueden distribuir a lo largo del procedimiento requerido}

Por lo que entonces

\begin{align*}
	a_{0} &= \dfrac{1}{3}\\
	a_n &= \dfrac{3}{\pi^2 n^2}\left[1-\cos\left(\frac{2n\pi}{3}\right)\right]\\
	b_n &= 0
\end{align*}





}

\shortsolution{
\textbf{Respuesta:}

\begin{align*}
	a_{0} &= \dfrac{1}{3}\\
	a_n &= \dfrac{3}{\pi^2 n^2}\left[1-\cos\left(\frac{2n\pi}{3}\right)\right]\\
	b_n &= 0
\end{align*}

}

\end{subpregunta}


