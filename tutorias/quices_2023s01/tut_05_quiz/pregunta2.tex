%% Quiz 5 Tutoría S12023
%% William Quirós Solano	

\Pregunta{5}{Dado el siguiente circuito}

\begin{center}
	\includegraphics[scale=1]{p02}
\end{center}

donde la fuente $v_{s}(t)$ está definida por

\begin{align*}
	v_{s}(t)=4+\dfrac{2}{\pi}\sum_{n=1}^{\infty}\dfrac{1}{n}\cos(nt)
\end{align*}

Determine la serie trigonométrica de Fourier en su representación de Amplitud-Fase para la tensión de salida $v_{o}(t)$.

\solution{
\textbf{Solución:}

Para el valor CD de la fuente correspondiente a $4\,$V dado que el condensador se comporta como un abierto y la bobina como un corto, la tensión de salida será

\begin{align*}
	V_o=4\,V\rightarrow\calif{1\,pto}
\end{align*}

Por otro lado, para el análisis en CA se tiene que la fuente tiene el fasor

\begin{align*}
	\mathbf{V_s}=\dfrac{2}{n\pi}\angle{0^{\circ}}\rightarrow\calif{1\,pto}
\end{align*}

Y considerando la frecuencia de operación $\omega_{0}=1\,rad/s$, entonces las impedancias de la bobina y el condensador son

\begin{align*}
	\mathbf{Z_L}=jn\\
	\mathbf{Z_C}=\dfrac{1}{jn}
\end{align*}

Y dado que la fuente está en paralelo con el resistor, el fasor de tensión de salida se puede obtener con un divisor de tensión dado por

\begin{align*}
	\mathbf{V_o} &= \dfrac{\mathbf{V_s \mathbf{Z_C}}}{\mathbf{Z_C}+\mathbf{Z_L}}\\
	\mathbf{V_o} &= \dfrac{\dfrac{2}{n\pi}\cdot\dfrac{1}{jn}}{\dfrac{1}{jn}+jn}\cdot\dfrac{jn}{jn}\\
	\mathbf{V_o} &= \dfrac{2}{\pi n}\cdot\dfrac{1}{1+(jn)^2}\\
	\mathbf{V_o} &= \dfrac{2}{\pi n (1-n^2)} \rightarrow\calif{2\,ptos}
\end{align*}

Con lo anterior y el valor CD entonces la serie para la tensión de salida $v_{o}(t)$ es

\begin{align*}
v_{o}(t)=4-\dfrac{2}{\pi}\sum_{n=1}^{\infty}\dfrac{1}{n(n^2-1)}\cos(nt)\,V\rightarrow\calif{1\,pto}
\end{align*}

}

\shortsolution{
\textbf{Respuesta:}

\begin{align*}
v_{o}(t)=4-\dfrac{2}{\pi}\sum_{n=1}^{\infty}\dfrac{1}{n(n^2-1)}\cos(nt)\,V
\end{align*}

}



