\documentclass[12pt,oneside,letterpaper, landscape]{article}

\usepackage[spanish]{babel}
\usepackage[utf8]{inputenc}
\usepackage{ifthen}                     % provide if-then-else operators
\usepackage{amsmath}
\usepackage{amssymb,amstext}            % AMS-math and symbols package

\usepackage{ftcap}                      % switch \abovecaptionskip and
                                        % \belowcaptionskip for tables, in 
                                        % order to avoid the caption to be
                                        % too near to the table itself
\usepackage{booktabs}                   % book type tabulars
\usepackage{rotating}
\usepackage{tabularx}


\usepackage{enumerate}
\usepackage{setspace}
\usepackage{array}
\usepackage{longtable}
\usepackage[dvipsnames,table]{xcolor}

%
% page layout
%
\setlength{\oddsidemargin}{0cm}         % Margin for odd numbered pages
\setlength{\evensidemargin}{0cm}        % Margin for even numbered pages
\addtolength{\topmargin}{-1.5cm}        % space between top and head
\addtolength{\headsep}{0.25cm}          % space between head and text
\setlength{\textwidth}{228mm}           % text width
\setlength{\textheight}{155mm}          % text height
\setlength{\headheight}{47pt}         % fancy headers wanted this
\parindent0em                           % indentation width of first line
\setlength{\headsep}{15pt}
\setlength{\topsep}{0pt}
\setlength{\itemsep}{0pt}


\usepackage{fancyhdr}
\pagestyle{fancy}
\lhead[]{
\footnotesize{Instituto Tecnológico de Costa Rica\\
Escuela de Ingeniería Electrónica\\
EL\,2114 Circuitos Eléctricos en Corriente Alterna
}}
\chead[]{}
\rhead[]{\footnotesize{I Semestre 2024}}
\lfoot[]{}
\cfoot[]{}
\rfoot[]{\thepage}
\renewcommand{\headrulewidth}{1pt}
\renewcommand{\footrulewidth}{0pt}

\begin{document}

\begin{center}
\large \textbf{Cronograma para Inicio del Programa de Tutorías CA-IS2024}
\end{center}

\medskip

\hspace{7mm} Para el programa de tutorías del IS2024 se ofrecerán 5 grupos presenciales de tutorías para estudiantes de Cartago y un grupo presencial para San Carlos. Para definir el inicio de las tutorías se seguirá el siguiente cronograma de actividades:

\begin{center}

\begin{longtable}{|>{\centering\arraybackslash}m{10mm}|m{85mm}|>{\centering\arraybackslash}m{35mm}|>{\centering\arraybackslash}m{40mm}|>{\centering\arraybackslash}m{35mm}|}
\hline 
\rowcolor{cyan}$\mathbf{N^{o}}$ 
& \centering \textbf{Actividad} 
& \textbf{Fecha} 
& \textbf{Responsable} 
& \textbf{Estado} \\
\hline 
1 
& Revisar el contrato a utilizar para ingresar al programa de tutorías. 
& 5-7 febrero 
& Cada profesor del curso 
& \textcolor{green!45!black}{En proceso} \\ 
\hline 
2 
& Dar lectura y explicación del programa del curso a los estudiantes en la primera clase. Explicar todo lo relacionado a las tutorías. 
&  7 febrero 
& Cada profesor del curso 
& \textcolor{red!75!black}{Pendiente} \\ 
\hline 
3 
& Solicitar a los tutores al menos 3 horarios disponibles para impartir sus tutorías respectivas. 
& 8 - 9 de febrero
& José Miguel 
& \textcolor{red!75!black}{Pendiente} \\ 
\hline 
4 
& Elaborar la primera encuesta con todos los horarios disponibles para abrir un grupo.   
& 8 de febrero  
& José Miguel 
& \textcolor{red!75!black}{Pendiente} \\ 
\hline 
5 
& Publicar la encuesta para seleccionar los horarios de los grupos de tutorías. 
& 8 - 9 de febrero  
& José Miguel 
& \textcolor{red!75!black}{Pendiente} \\ 
\hline 
6
& Seleccionar los horarios de los 5 grupos de tutorías.
& 10 de febrero  
& José Miguel 
& \textcolor{red!75!black}{Pendiente} \\ 
\hline 
7
& Publicar la encuesta para matrícula a los grupos de tutorías disponibles según la selección de horarios realizada previamente. 
& 10-12 de febrero 
& José Miguel 
& \textcolor{red!75!black}{Pendiente} \\ 
\hline 
8 
& Publicar las listas finales de los 5 grupos de tutorías 
& 13 febrero
& José Miguel 
& \textcolor{red!75!black}{Pendiente} \\ 
\hline 
9 
& Proceso de firma de contratos*  
& 14 - 16 de febrero 
& Cada profesor con su grupo respectivo
& \textcolor{red!75!black}{Pendiente} \\ 
\hline 
10 
& Inicio de las tutorías
& Semana 3 lectiva 
& - 
& \textcolor{red!75!black}{Pendiente} \\ 
\hline 
11 
& Cierre de las tutorías 
& Semana 17 lectiva 
& - 
& \textcolor{red!75!black}{Pendiente} \\  
\hline 
\end{longtable} 
\end{center}


*\underline{Proceso de firma de contratos}:  

\begin{itemize}
    \item Realizar las impresiones de los contratos durante la primera semana lectiva. Preparar dos copias por estudiante (encargado en Cartago: JMBR).
    \item Poner el sello de la Escuela a cada contrato (encargado en Cartago: JMBR). 
    \item Entregar los contratos a cada profesor para proceder a la firma de cada uno de ellos durante la semana 2 lectiva (encargado en Cartago: JMBR).
    \item Todo estudiante debe firmar el contrato, tanto aquellos que decidan entrar al programa de tutorías y los que decidan que no. Es importante ubicar los estudiantes que no hayan firmado el contrato para pedir que procedan a la firma.
\end{itemize}  

\end{document}
