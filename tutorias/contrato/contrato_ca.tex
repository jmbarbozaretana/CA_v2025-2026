\documentclass[12pt,oneside,letterpaper]{article}

\usepackage[spanish]{babel}
\usepackage[utf8]{inputenc}
\usepackage{ifthen}                     % provide if-then-else operators
\usepackage{amsmath}
\usepackage{amssymb,amstext}            % AMS-math and symbols package

\usepackage{ftcap}                      % switch \abovecaptionskip and
                                        % \belowcaptionskip for tables, in 
                                        % order to avoid the caption to be
                                        % too near to the table itself
\usepackage{booktabs}                   % book type tabulars
\usepackage{rotating}
\usepackage{tabularx}


\usepackage{enumerate}
\usepackage{setspace}
\usepackage[dvipsnames,table]{xcolor}

%
% page layout
%
\setlength{\oddsidemargin}{0cm}         % Margin for odd numbered pages
\setlength{\evensidemargin}{0cm}        % Margin for even numbered pages
\addtolength{\topmargin}{-1.5cm}        % space between top and head
\addtolength{\headsep}{0.25cm}          % space between head and text
\setlength{\textwidth}{165mm}           % text width
\setlength{\textheight}{215mm}          % text height
\setlength{\headheight}{35pt}         % fancy headers wanted this
\parindent0em                           % indentation width of first line
\setlength{\headsep}{15pt}
\setlength{\topsep}{0pt}
\setlength{\itemsep}{0pt}


\usepackage{fancyhdr}
\pagestyle{fancy}
\lhead[]{
\footnotesize{Escuela de Ingeniería Electrónica\\
Departamento de Orientanción y Psicología\\
Propuesta de evaluación alternativa para el cuso de EL2114 Circuitos Eléctricos en Corriente Alterna
}}
\chead[]{}
\rhead[]{}
\lfoot[]{}
\cfoot[]{}
\rfoot[]{\thepage}
\renewcommand{\headrulewidth}{1pt}
\renewcommand{\footrulewidth}{0pt}

\begin{document}

\begin{center}
\textbf{Contrato de aplicación del artículo 58 del Reglamento del Régimen  Enseñanza-Aprendizaje del Tecnológico de Costa Rica y sus Reformas}
\end{center}

\bigskip
En consideración a lo estipulado en el artículo 58 del Reglamento del Régimen Enseñanza-Aprendizaje del Tecnológico de Costa Rica y sus Reformas\footnote{%
Artículo 58\\ 
La persona estudiante que pierda por segunda vez una asignatura:
\begin{enumerate}[a.]
\item Cuando matricule de nuevo la asignatura, el número total de créditos no podría ser mayor de doce, incluyendo los del curso reprobado. Vía excepción le corresponde al Director de Escuela resolver de acuerdo con los criterios establecidos por el Consejo de Escuela, en aquellos casos en que se solicite.
\item La escuela o departamento académico encargado de la carrera asignará al estudiante un tutor, cuya función principal será asesorarlo en cuanto a la matrícula y hacer un seguimiento de su trabajo académico. Además, se debe contar con opciones académicas diferentes a fin de posibilitar el éxito del estudiante. Por ejemplo, técnicas de estudio, cursos SIP, cursos especiales, etc.
\end{enumerate}

\textsl{Artículo modificado por el Consejo Institucional en la sesión extraordinaria No 3176 celebrada el 24 de abril del 2020.}
}
se ofrecerá a las personas estudiantes en condición R2 o más, matriculados en el curso EL\,2114 Circuitos Eléctricos en Corriente Alterna, durante el \textit{segundo semestre de 2025}, un sistema de evaluación diferenciada, con el objetivo de mejorar el aprovechamiento académico del curso. Además, dicha evaluación diferenciada se ofrecerá también a las personas estudiantes R0 y R1 que voluntariamente decidan acogerse a la misma. La anterior iniciativa responde al bajo rendimiento académico que se ha presentado en los semestres anteriores en este curso.


\bigskip
\textbf{Objetivos}

\begin{itemize}\itemsep0pt
	\item Mejorar los niveles de promoción mostrados en el curso.
	\item Favorecer el desarrollo de estrategias para aprender a aprender.
	\item Estimular el aprendizaje cooperativo. 
	\item Fortalecer en las personas estudiantes hábitos y estrategias de estudio.
\end{itemize}

\textbf{Evaluación del curso}

\medskip

\hspace{7mm} La propuesta de evaluación del curso sigue incluyendo los mismos rubros contemplados en el programa, a saber: tres exámenes parciales. Estas evaluaciones corresponderán al 90\% de la nota final del curso. El restante 10\% corresponderá a las actividades de evaluación definidas para el programa de tutorías.



%\bigskip
\clearpage

\textbf{Tutorías} 

\medskip

\hspace{7mm}El programa de tutorías estará integrado por una serie de actividades académicas que se distribuyen a lo largo de todo el semestre. Así, dichas actividades serán:

\begin{enumerate}[a)]\itemsep0pt
\item \underline{Tareas}: serán algunos ejercicios que se asignarán a las personas estudiantes previamente a cada taller de estudio, los cuales deberán resolver y entregar al inicio  de cada taller. La entrega debe ser en papel, escrito a mano y será un requisito para validar la participación en dicho taller. 

\item \underline{Talleres de estudio}: Son sesiones presenciales en las cuales la persona tutora trabajará con las personas estudiantes para analizar los ejercicios de la tarea, visualizar errores posibles, aclarar dudas, etc. Además, la persona tutora podrá trabajar otros ejercicios adicionales que formen parte de las guías de ejercicios de las tutorías u otras guías adicionales.

\item \underline{Ejercicios finales de comprobación:} al finalizar cada taller la persona tutora compartirá un ejercicio final para evaluar la comprensión de los  temas abordados durante cada taller. Cada persona estudiante deberá realizar este ejercicio de manera individual y a mano (no por medios digitales), para ser entregado inmediatamente al finalizar el taller. También serán un requisito para validar la participación de dicho taller. 

%se aplicarán como evaluación para medir el dominio de cada estudiante sobre el contenido respectivo asignado. Estas pruebas serán presenciales en horarios y fechas que se establecen más adelante en este documento.

\end{enumerate}


%\medskip
 
\textbf{Instrucciones generales}

\begin{itemize}

\item Las personas estudiantes que participen del programa de tutorías estarán organizados en grupos con un horario fijo y estarán a cargo de uno de los tutores del curso. Además, tanto los grupos y sus horarios serán organizados y definidos por los profesores del curso de forma previa a la firma de este contrato.

\item Los ejercicios que se asignarán como tarea para entregar en cada taller de estudio serán tomados de las guías de ejercicios dispuestos para la tutoría. Estas guías estarán disponibles en el tecDigital y serán publicadas conforme se requieran a lo largo del semestre. Los ejercicios finales de comprobación no se conocerán hasta cada taller.

\item Para determinar si la tarea y el ejercicio final de comprobación, entregados por la persona estudiante, están completos, no se considerará que los resultados obtenidos sean correctos, sino se tomará para ello el grado de análisis, profundidad y completitud de las respuestas. Por lo tanto, bajo este criterio, la persona tutora deberá definir si una tarea o ejercicio final está o no completo. Si una tarea o ejercicio final no están completos, equivale a que la persona estudiante se considere como ausente en dicho taller de estudio, esto aunque haya participado presencialmente del mismo. Es decir, para contabilizar la participación del taller la persona estudiante debe tener la tarea y el ejercicio final completos para dicho taller.

%\item Al abarcar los mismos temas estudiados ese día, los ejercicios finales de comprensión estarán diseñados para que cualquier persona estudiante que haya seguido el taller a consciencia  pueda realizarlos.

\item Si un estudiante se ausenta a un taller de estudio y tiene una justificación médica o académica válida emitida por un profesional calificado, tiene un plazo de 3 días hábiles posteriores a la sesión presencial para presentarla. La justificación deberá ser presentada al profesor del curso respectivo y al mismo tiempo, el profesor recibirá la tarea que la persona estudiante tenía que entregar en el taller de estudio que se ausentó. En el caso de los ejercicios de comprobación, se deberá coordinar con cada profesor para ser realizado.
%En el caso de los quices fuciona de la misma forma y en caso de ser aceptada la justificación por el profesor a cargo, se buscará la reposición de la evaluación en un horario y lugar adecuado.

\item La persona estudiante asignada como tutor a cada grupo apoyará a las personas estudiantes en la atención de consultas, el análisis de errores, retroalimentación sobre soluciones planteadas, mostrar estrategias alternativas en la solución de problemas, etc. 

\item Cada taller de estudio se desarrollará de forma presencial en el horario y lugar asignado a cada grupo y tendrá una duración máxima de 2 horas. Alrededor de 20 minutos de este tiempo serán para la realización del ejercicio final de comprobación.

%\item Cada quiz de comprobación representará un instrumento de medición del aprendizaje individual de cada estudiante. Con ello, cada quiz estará diseñado en función de los ejercicios que forman parte de las guías utilizadas en el programa de tutorías.

%\item Cada uno de los quices se aplicarán a las 3\,pm los días lunes en el lugar o aula que se indique previamente. Todas las personas estudiantes realizarán cada quiz en este horario, esto sin importar que el horario del taller de estudio sea otro distinto. La calendarización específica de cada quiz se puede observar más abajo en este documento.


\end{itemize}

%\clearpage 
 
\textbf{Cronograma}

\medskip

\hspace{7mm} El cronograma respectivo del programa de tutorías se define en la siguiente tabla:

\bigskip
\begin{center}
\begin{tabular}{|>{\centering\arraybackslash}m{15mm}|m{45mm}|>{\centering\arraybackslash}m{30mm}|}
\hline 
\textbf{Semana Lectiva} 
& \centering \textbf{Fecha} 
& \textbf{Talleres de estudio}\\
\hline 
1 & 04 de ago - 08 de ago & - \\ 
\hline 
2 & 11 de ago - 15 de ago & -\\ 
\hline 
3 & 18 de ago - 22 de ago & N\textsuperscript{o}1 \\ 
\hline 
4 & 25 de ago - 29 de ago & N\textsuperscript{o}2 \\ 
\hline
5 & 01 de set - 05 de set & N\textsuperscript{o}3 \\ 
\hline 
6 & 08 de set - 12 de set & N\textsuperscript{o}4 \\ 
\hline 
7 & 15 de set - 19 de set & N\textsuperscript{o}5 \\ 
\hline
8 & 22 de set - 26 de set & N\textsuperscript{o}6 \\ 
\hline
9 & 29 de set - 03 de oct & N\textsuperscript{o}7 \\
\hline 
10 & 06 de oct - 10 de oct & N\textsuperscript{o}8 \\ 
\hline 
11 & 13 de oct - 17 de oct & N\textsuperscript{o}9 \\ 
\hline 
12 & 20 de oct - 24 de oct & N\textsuperscript{o}10 \\ 
\hline 
13 & 27 de oct - 31 de oct & N\textsuperscript{o}11 \\ 
\hline 
14 & 03 de nov - 07 de nov & N\textsuperscript{o}12 \\ 
\hline 
15 & 10 de nov - 14 de nov & N\textsuperscript{o}13 \\ 
\hline 
16 & 17 de nov - 21 de nov & N\textsuperscript{o}14 \\  
\hline
\end{tabular} 
\end{center}

\newpage

\textbf{Obligaciones y/o condiciones}

\medskip

\hspace{7mm}Para obtener el porcentaje correspondiente al rubro de tutoría en la nota del curso las personas estudiantes estarán en la obligación de cumplir y considerar lo siguiente:


\begin{enumerate}[1.]\itemsep0pt

\item La propuesta de evaluación del curso seguirá incluyendo los mismos rubros contemplados en el programa, a saber: tres exámenes parciales. Estas evaluaciones corresponderán al 90\% de la nota final del curso, donde cada uno de los tres exámenes parciales tendrán un valor de 30\%. El restante 10\% corresponderá para las actividades de evaluación definidas en el programa de tutorías.


\item El programa de tutorías tendrá un total de 14 talleres de estudio presenciales. En cada uno de los talleres de estudio la persona estudiante deberá entregar la tarea asignada al iniciar el taller y realizar el respectivo ejercicio de comprobación, el cual se realizará al final de la sesión. Esto implica que habrán 14 tareas y 14 ejercicios de comprobación en total.

\item La asistencia a un taller tendrá un valor de 1\%. El valor final del rubro final de tutorías será proporcional a la cantidad de talleres asistidos, para un porcentaje máximo de 10\%.  Es decir, cada persona estudiante tendrá la opción de ausentarse a mínimo 4 talleres y aún mantener la posibilidad de tener el porcentaje máximo de 10\%.
%\item La persona estudiante deberá asistir al menos al 85\% de las lecciones del curso. De lo contrario, se le asignará una nota de 0 en el rubro de tutorías.



%\item La persona estudiante deberá asistir al menos al 85\% de las lecciones del curso. De lo contrario, se le asignará una nota de 0 en el rubro de tutorías.

%\item La persona estudiante deberá asistir presencialmente a los 15 talleres de estudio que se impartirán de forma presencial durante el semestre. Aquella persona que se ausente en más de 3 ocasiones a los talleres de estudio, perderá automáticamente el 10\% asignado al rubro de tutorías en la evaluación del curso.

\item Los ejercicios que formen parte de cada tarea se deberán asignar a las personas estudiantes con al menos 8 días naturales de anticipación por parte del profesor responsable. Así, las personas estudiantes deberán trabajar la solución de dichos ejercicios para entregarlos en el taller de estudio indicado. 

\item Tanto la tarea como el ejercicio de comprobación deberán ser escritos en papel y a mano; además, deberán ser entregados al tutor respectivo justo al inicio y justo al final de cada taller de estudio, respectivamente. La tarea no se puede entregar ni durante ni al final del taller. Para el ejercicio de comprobación se dará un tiempo prudencial para su realización al final del taller (alrededor de 20min) y deberá ser entregado inmediatamente terminado este tiempo.

%\item Cada quiz tendrá un valor de un tercio del valor total del 10\% que se debe asignar en la nota del curso debido a la participación en el programa de tutorías.

\item La participación en esta modalidad de evaluación es voluntaria, pero una vez que la persona estudiante haya manifestado su consentimiento a las nuevas condiciones de evaluación y luego de vencido el periodo estipulado por la cláusula de rescisión, no será posible regresar al sistema de evaluación anterior definido en la carta al estudiante del curso respectivo.

\item \textbf{\underline{CLÁUSULA DE RESCISIÓN}}. Si la persona estudiante considera que la evaluación especificada en este contrato no le favorece, podrá presentar su solicitud de rescisión por escrito a más tardar en la sexta semana lectiva del semestre en curso. De rescindir el contrato, regirán para la persona estudiante los criterios de evaluación definidos en el programa del curso para el semestre respectivo.

\item La persona estudiante queda protegida por la ley 8968, en relación con la protección de la persona frente al tratamiento de sus datos personales.

\end{enumerate}

%\clearpage
\textbf{En fe de lo anterior, se firma contrato por duplicado en la ciudad de Cartago.}

\medskip
\spacing {1.5}
Yo \rule{80mm}{0.1mm}, estudiante activo/a de la carrera de Ingeniería
\rule{35mm}{0.1mm}, carné \rule{35mm}{0.1mm}, teléfono celular \rule{35mm}{0.1mm}, correo-e 
\rule{75mm}{0.1mm}, estoy de acuerdo con lo propuesto en este documento y Sí (   ) No (   ) acepto se me aplique esta propuesta de evaluación con los beneficios y obligaciones estipuladas.
   
\bigskip
\bigskip

Fecha: \rule{30mm}{0.1mm}\hspace{30mm}Firma del estudiante: \rule{52mm}{0.1mm}

\vspace{5mm}
Nombre del docente: \rule{65mm}{0.1mm}\hspace{5mm}
%Nombre del docente: \underline{Laura Cabrera Quirós}
%Nombre del docente: \underline{José Miguel Barboza Retana}
%Nombre del docente: \underline{William Quirós Solano}
%Nombre del docente: \underline{Saúl Guadamuz Brenes}

\vspace{5mm}
	 
Firma del docente: \rule{68mm}{0.1mm} \hspace{10mm} Sello de la Escuela

\end{document}
