%% 2do Examen parcial S2 2022
%% Roberto Molina Robles

\Pregunta{2}{Un ingeniero construye para un cliente un filtro de tipo Chebyshev en un amplificador de telecomunicaciones utilizando el siguiente circuito:}

\begin{center}
\includegraphics[scale=1]{p08}
\end{center}

El filtro solicitado es de tipo $pasaaltas$ con una frecuencia de corte de $10\,$MHz. Para ello, el ingeniero compró los siguientes componentes: $R=1\,$k$\Omega$, $C=150\,$pF, $L_1=150\,\mu$H y $L_2=300\,\mu$H, e implementó el circuito. No obstante, al hacer mediciones sobre el circuito descubrió que por un error de cálculo la frecuencia de corte se encuentra en $6.28\,$Mrad/s. Con el presupuesto restante, ya no es posible conseguir nuevos inductores, por lo que solo se puede comprar una nueva resistencia y un capacitor para corregir el error. Bajo estas condiciones, ¿qué valores debería tener los nuevos $R$ y $C$ para alcanzar la frecuencia de corte deseada de $10\,$MHz?


\abcdE
{$R=1.59\,$k$\Omega$ y $C=59.16\,$pF\medskip}
{$R=100\,\Omega$ y $C=15\,$nF\medskip}
{$R=10\,$k$\Omega$ y $C=379.22\,$nF\medskip}
{$R=1.59\,$k$\Omega$ y $C=1.5\,$pF\medskip}
{$R=10\,$k$\Omega$ y $C=1.5\,$pF\medskip}