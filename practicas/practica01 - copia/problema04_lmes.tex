%% 2do Examen parcial S1 2022
%% José Miguel Barboza Retana 

\Problema{6}{Respusta en Frecuencia: Diagramas de Bode}

\vspace{2mm}

Considere la siguiente función de transferencia encargada de describir el comportamiento en frecuencia de un circuito eléctrico desconocido.

\begin{equation*}
\mathbf{H(\omega)} = \dfrac{25\left(20+2j\omega\right)}{\left(25+\dfrac{j\omega}{4}\right)\left(1+j\omega\right)}
\end{equation*}



\begin{subpunto}
    \item Determine la expresión de la función de magnitud $|\mathbf{H(\omega)}|_{dB}=20\log_{10}(|\mathbf{H(\omega)}|)$. Exprese la misma en forma expandida identificando cada uno de los factores según la conocida forma estándar. \partialPoints{1}    
    \item Determine la expresión de la función de fase $\Phi(\omega)=\angle{\mathbf{H(\omega)}}$. Exprese la misma en forma expandida identificando cada uno de los factores según la conocida forma estándar.\partialPoints{1}
    \item Grafique el diagrama asintótico de Bode (magnitud y fase) de la función $\mathbf{H(\omega)}$ utilizando un papel semilogarítmico adecuado. \partialPoints{4}              
\end{subpunto}

