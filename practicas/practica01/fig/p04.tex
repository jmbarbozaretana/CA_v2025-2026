\documentclass[class=article,border=10pt]{standalone}
\usepackage{array,tabularx}
\usepackage{float}
\usepackage{graphicx}
\usepackage{verbatim}
\usepackage{textcomp}
\usepackage{amssymb, amsmath, amsbsy}
\usepackage{tikz}
\usepackage{pgfplots}
\usetikzlibrary{shapes,arrows}
\usepackage[europeanresistors,americaninductors,RPvoltages]{circuitikz}
\pgfplotsset{width=10cm,compat=1.9}

\begin{document}
%Circuito
\begin{circuitikz} [american, scale=1]
    \draw
    (0,0) to [V,v=$v_{s}(t)$] (0,3)
    (0,3) to [R, l=$1\,\Omega$] (3,3)
    (4,3) to [R, l=$1\,\Omega$] (4,0)
    (4,5) to [R, l=$2\,\Omega$] (7,5)
    (4,3) to [L, l=$1\,$H] (7,3)
    (7,3) to [C, l_=$1\,$F] (7,0)
    (10,3) to [R, l=$1\,\Omega$] (10,0)
    (0,0) to [short,-] (10,0) 
    (7,3) to [short,-] (10,3)
    (4,3) to [short,-] (4,5)
    (7,3) to [short,-] (7,5)
    (3,3) to [short,-] (4,3)
;

\draw[dashed,line width=1pt,blue] (3,-0.5) -- (8,-0.5);
\draw[dashed,line width=1pt,blue] (3,-0.5) -- (3,6);
\draw[dashed,line width=1pt,blue] (3,6) -- (8,6);
\draw[dashed,line width=1pt,blue] (8,6) -- (8,-0.5);



\draw[line width=1pt] (3,3) to [short,-* ] (3,3);
\draw[line width=1pt] (3,0) to [short,-* ] (3,0);
\draw[line width=1pt] (8,3) to [short,-* ] (8,3);
\draw[line width=1pt] (8,0) to [short,-* ] (8,0);


 \draw(2.5,3) to [open, v=$v_1(t)$] (2.5,0);
  \draw(8.5,3) to [open, v=$v_2(t)$] (8.5,0);
  
  \draw (3.85,-1) -- (3.85,-1) node[right] {$\textbf{Red de 2 Puertos}$};

\end{circuitikz}

\end{document}


