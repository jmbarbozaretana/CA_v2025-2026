\documentclass[12pt,oneside,letterpaper, landscape]{article}

\usepackage[spanish]{babel}
\usepackage[utf8]{inputenc}
\usepackage{ifthen}                     % provide if-then-else operators
\usepackage{amsmath}
\usepackage{amssymb,amstext}            % AMS-math and symbols package

\usepackage{ftcap}                      % switch \abovecaptionskip and
                                        % \belowcaptionskip for tables, in 
                                        % order to avoid the caption to be
                                        % too near to the table itself
\usepackage{booktabs}                   % book type tabulars
\usepackage{rotating}
\usepackage{tabularx}
\usepackage{multicol}


\usepackage{enumerate}
\usepackage{setspace}
\usepackage{array}
\usepackage{longtable}
\usepackage[dvipsnames,table]{xcolor}

%
% page layout
%
\setlength{\oddsidemargin}{0cm}         % Margin for odd numbered pages
\setlength{\evensidemargin}{0cm}        % Margin for even numbered pages
\addtolength{\topmargin}{-1.5cm}        % space between top and head
\addtolength{\headsep}{0.25cm}          % space between head and text
\setlength{\textwidth}{228mm}           % text width
\setlength{\textheight}{155mm}          % text height
\setlength{\headheight}{47pt}         % fancy headers wanted this
\parindent0em                           % indentation width of first line
\setlength{\headsep}{15pt}
\setlength{\topsep}{0pt}
\setlength{\itemsep}{0pt}


\usepackage{fancyhdr}
\pagestyle{fancy}
\lhead[]{
\footnotesize{Instituto Tecnológico de Costa Rica\\
Escuela de Ingeniería Electrónica\\
EL-2114 Circuitos Eléctricos en Corriente Alterna
}}
\chead[]{}
\rhead[]{\footnotesize{II Semestre 2025}}
\lfoot[]{}
\cfoot[]{}
\rfoot[]{\thepage}
\renewcommand{\headrulewidth}{1pt}
\renewcommand{\footrulewidth}{0pt}

%%%%%%%%%%%%%%%%%%%%%%%%%%%%%%%%%%%%%%%%%%%%%%%%%%%%%%%%%%%%%%%%%%%%%%%%%%%%%%%%%%%%%%%
%%%%%%%%%%%%%%%%%%%%%%%%%%%%%%%%%%%%%%%%%%%%%%%%%%%%%%%%%%%%%%%%%%%%%%%%%%%%%%%%%%%%%%%

\begin{document}

\begin{center}
\textbf{\underline{\Large{Primer Parcial CA-IIS2025}}}
\end{center}

\bigskip

\begin{multicols}{2}
\underline{\textbf{\large{Programación ordinaria:}}}   
\begin{itemize}
	\item \textbf{Fecha de aplicación:} \textcolor{red}{sábado 13 de setiembre 2025}
    \item \textbf{Hora de inicio:} \textcolor{red}{8 am}
    \item \textbf{Duración:} \textcolor{red}{3 horas}
    \item \textbf{Modalidad}: \textcolor{red}{presencial}
\end{itemize}
    
\underline{\textbf{\large{Programación extraordinaria:}}}
\begin{itemize}
	\item \textbf{Fecha de aplicación:} \textcolor{red}{martes 16 de setiembre 2025}
    \item \textbf{Hora de inicio:} \textcolor{red}{8 am}
    \item \textbf{Duración:} \textcolor{red}{3 horas}
    \item \textbf{Modalidad}: \textcolor{red}{presencial}
\end{itemize}

\end{multicols}

\bigskip

\textbf{\large{Materiales permitidos para los estudiantes durante la aplicación del examen:}}
\begin{itemize}
    \item No se permite ningún tipo de calculadora electrónica programable.
    \item Formulario oficial del curso impreso o fotocopiado y no puede tener ningún tipo de anotación adicional. 
    \item Accesorios de oficina: lapiceros, lápiz de escribir, lapices de colores, etc.
    \item Hojas blancas/rayadas/cuadriculadas o cuaderno de examen.
\end{itemize}


\bigskip

\textbf{\large{Aulas reservadas:}}
\begin{itemize}
    \item Grupo 01. Cartago. Laura Cabrera Quirós. Aula: \textcolor{red}{K1-418}
    \item Grupo 02. Cartago. José Miguel Barboza Retana. Aula: \textcolor{red}{K1-518}
	\item Grupo 03. Cartago. William Quirós Solano. Aula: \textcolor{red}{K1-211}   
    \item Grupo 50. San Carlos. Saúl Guadamuz Brenes. Aula: \textcolor{red}{Por definir}
\end{itemize}


\clearpage

\large{\textbf{Aspectos importantes:}}

\normalsize

\begin{itemize}
\item La prueba deberá ser resuelta por los estudiantes en forma individual, clara y ordenada. 

\item El estudiante que llegue después de 30min de iniciada la prueba no tendrá derecho de realizar el examen.

\item El examen estará conformado por preguntas de desarrollo entre 5 a 10 pts para cada pregunta.

\item \textcolor{blue}{En el diseño de las preguntas de desarrollo, plantear el ejercicio de forma que no tenga más de 3 o 4 preguntas o ítemes, esto con el fin de evitar problemas muy extensos}.

\end{itemize}


%%%%%%%%%%%%%%%%%%%%%%%%%%%%%%%%%%%%%%%%%%%%%%%%%%%%%%%%%%%%%%%%%%%%%%%%%%%%%%%%%%%%%%%
%%%%%%%%%%%%%%%%%%%%%%%%%%%%%%%%%%%%%%%%%%%%%%%%%%%%%%%%%%%%%%%%%%%%%%%%%%%%%%%%%%%%%%%

\definecolor{cssgreen}{rgb}{0.0, 0.5, 0.0}
\definecolor{yellow2}{rgb}{1,0.55,0.0}
\definecolor{ligthgray}{rgb}{1,0.55,0.0}

\vspace{10mm}

\newcommand{\listo}{\textcolor{cssgreen}{Listo}}
\newcommand{\pendiente}{\textcolor{red}{\textbf{Pendiente}}}

\textbf{\Large{Cronograma de actividades:}}

\begin{center}
\begin{longtable}{|>{\centering\arraybackslash}m{10mm}|m{120mm}|>{\centering\arraybackslash}m{50mm}|>{\centering\arraybackslash}m{25mm}|}
\hline 
\rowcolor{cyan}$\mathbf{N^{o}}$ 
& \centering \textbf{Actividad} 
& \textbf{Fecha} 
& \textbf{Estado}\\
\hline 
1 
& Asignación de las preguntas por diseñar para cada profesor. 
& \textcolor{black}{26 de agosto 2025}  
& \listo\\ 
\hline 

2 
& Desarrollo de cada una de los ejercicios asignados. Entregar el ejercicio con solución completa y puntajes respectivos.
& \textcolor{black}{26 ago - 8 set 2025} & \pendiente\\ 
\hline

3 
& Revisión de cada pregunta del examen según distribución propuesta más adelante. Recordar que la revisión implica hacer un análisis muy detallado de: la redacción de la pregunta, consistencia de los datos de la pregunta, de la respuesta correcta, de las respuestas que no son correctas, asignación de puntos, etc. Este es un paso muy importante que no se debe hacer a la carrera. 
& \textcolor{black}{9 setiembre 2025} 
& \pendiente\\ 
\hline 

4 
& Impresión de los exámenes
& \textcolor{black}{10 - 12 setiembre 2025}
& \pendiente\\ 
\hline

5 
& \textcolor{magenta}{\textbf{Aplicación de la prueba}}
& \textcolor{magenta}{\textbf{13 de setiembre 2025}}
& \pendiente \\ 
\hline

7
& Revisión de los exámenes para definir los puntos obtenidos por estudiante y así calcular la curva.
& \textcolor{black}{13 - 26 setiembre 2025}
& \pendiente\\ 
\hline

\end{longtable} 
\end{center}

\clearpage


%%%%%%%%%%%%%%%%%%%%%%%%%%%%%%%%%%%%%%%%%%%%%%%%%%%%%%%%%%%%%%%%%%%%%%%%%%%%%%%%%%%%%%%
%%%%%%%%%%%%%%%%%%%%%%%%%%%%%%%%%%%%%%%%%%%%%%%%%%%%%%%%%%%%%%%%%%%%%%%%%%%%%%%%%%%%%%%

\textbf{\large{Propuesta detallada:}}

\begin{center}
\begin{longtable}{|m{26mm}|m{125mm}|m{25mm}|m{25mm}|}
\hline 
\rowcolor{white!50!blue}\centering \textbf{Ítem}  
& \centering\arraybackslash\textbf{Sugerencias}
& \centering \textbf{Diseñador} 
& \centering\arraybackslash\textbf{Revisor} \\

\hline 
\textbf{Problema 01} 
& \underline{Unidad}: 1 \newline \underline{Detalle}: Manejo de ondas y fasores. Adelantos, retrasos, desfases, suma de ondas, diagramas fasoriales, etc.
& Saúl
& William\\
\hline


\rowcolor{white!60!gray}\textbf{Problema 02}  
& \underline{Unidad}: 1 
\newline \underline{Detalle}: Análisis de circuitos en corriente alterna: fasores, ondas, etc. Este análisis es abierto respecto a que posibles técnicas se requieran utilizar para el mismo. Considerar quizás el uso de fuentes dependientes y/o amplificadores.
& José Miguel
& Saúl\\
\hline

\textbf{Problema 03}  
& \underline{Unidad}: 2 
\newline \underline{Detalle}: Análisis de potencia en alterna de algún circuito dado, el análisis puede implicar manejar magnitudes en valores pico o eficaces y llevar al estudiante por análisis de potencia instantánea, promedio, máxima transferencia de potencia promedio, compleja. Quizás un enfoque más a partir de un circuito en corriente alterna, lograr extraer información por análisis relaciona a la potencia en ca.
& Laura 
& José Miguel\\
\hline

\rowcolor{white!60!gray}\textbf{Problema 04} 
& \underline{Unidad}: 2 
\newline \underline{Detalle}: Análisis de potencia en alterna donde llevar al estudiante al manejo de la potencia compleja y todos sus componentes, considerar el concepto de conservación de potencia en ca, factor de potencia, corrección del factor de potencia. Quizás un enfoque más de a partir de información sobre el consumo de potencia de un sistema, analizar situaciones al respecto.
& William
& Laura\\
\hline
 
\end{longtable} 
\end{center}

\textcolor{blue}{NOTA IMPORTANTE: Las sugerencias indicadas en cada pregunta son con el fin de que se trate de evaluar cosas diferentes, no repetir cosas. Pero, pueden ser creativos en cada caso, no se limiten a lo que se anotó como detalle en el espacio de sugerencias de la pregunta.}\\

\end{document}