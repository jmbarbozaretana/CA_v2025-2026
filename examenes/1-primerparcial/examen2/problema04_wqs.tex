%%%%%%%%%%%%%%%
%% Problema4 %%
%% 2025 S2   %%
%% WQS       %%

\Problema{8}{Análisis de Potencia en CA}

\medskip

Dado el siguiente circuito

\begin{center}
\includegraphics[scale=1]{prob04_wqs}
\end{center}

Conociendo que
\begin{itemize}
\item La impedancia $\mathbf{Z}_1$ es inductiva y consume una potencia real de $7.5\,$kW y reactiva de $9\,$kVAR.
\item La impedancia $\mathbf{Z}_2$ es capacitiva y consume una potencia real de $2.1\,$kW y reactiva de $1.8\,$kVAR.
\item $\mathbf{V}_o=480\angle 0^{\circ}\,V_{rms}$
\end{itemize}

Responda lo siguiente:
\begin{subpunto}
\item Determine el factor de potencia visto por la fuente.\partialPoints{6}

\solution{
Calculando las potencias complejas
\begin{align*}
\mathbf{S_1}=7,5+j9\,kVA\\
\mathbf{S_2}=2.1-j1,8\,kVA\\
\mathbf{S_R}=\dfrac{480^2}{48}\\
\mathbf{S_L}=j\dfrac{480^2}{19,2}
\end{align*}
\calif{2 pts, 0.5 pts por cada potencia compleja correcta}

con esto, la potencia compleja en las impedancias en paralelo es
\begin{align*}
\mathbf{S_Z}=\mathbf{S_1}+\mathbf{S_2}+\mathbf{S_R}+\mathbf{S_L}\\
\mathbf{S_Z}=7500+j9000+2100-j1800+\dfrac{480^2}{48}+j\dfrac{480^2}{19,2}\\
\mathbf{S_Z}=24000\angle 53.13^{\circ}\,VA
\end{align*}
\calif{1 pto por la potencia compleja en las impedancias en paralelo}

y dado que la impedancia está en serie con la fuente y la bobina
\begin{align*}
\mathbf{S_Z}=\mathbf{V_o}\mathbf{I_s}^*\\
\mathbf{I_s}=\Big(\dfrac{\mathbf{S_Z}}{\mathbf{V_o}}\Big)^*=\Big(\dfrac{24000\angle 53.13^{\circ}}{480\angle 0^{\circ}}\Big)^*\\
\mathbf{I_s}=50\angle-53.13^{\circ}\,A_{rms}
\end{align*}
\calif{1 pto por la corriente en la fuente}

con esto entonces
\begin{align*}
\mathbf{S_{total}}=\mathbf{S_Z}+(50)^2j0,5\\
\mathbf{S_{total}}=25011,25\angle54.85^{\circ}\,VA
\end{align*}
\calif{1 pto por la potencia compleja total}

por lo que el factor de potencia visto por la fuente es
\begin{align*}
fp=\cos(54.85^{\circ})\downarrow
\end{align*}
\calif{1 pto por el valor correcto de el factor de potencia y su respectivo comportamiento}
}

\item El fasor de la fuente $\mathbf{V}_s$.\partialPoints{2}

\solution{
Sabiendo que la potencial total compleja está dada por
\begin{align*}
\mathbf{S_{total}}=\mathbf{V_s}\mathbf{I_s}^*
\end{align*}
entonces el fasor de la fuente se calcula con
\begin{align*}
\mathbf{V_s}=\dfrac{\mathbf{S_{total}}}{\mathbf{I_s}^*}\\
\mathbf{V_s}=\dfrac{25011,25\angle54.85^{\circ}}{50\angle53.13^{\circ}}\\
\mathbf{V_s}=500,22\angle1.72^{\circ}\,V_{rms}
\end{align*}
}
\end{subpunto}
