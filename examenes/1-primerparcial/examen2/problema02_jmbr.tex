%%%%%%%%%%%%%%%
%% Problema2 %%
%% 2025 S1   %%
%% JMBR      %%

\Problema{7}{Análisis de Circuitos en CA}

\medskip

Considere el circuito mostrado a continuación:

\begin{center}
  \includegraphics[scale=1]{prob02}
\end{center}

Determine la relación de desfase existente entre las ondas de corriente $i_x(t)$ e $i_L(t)$, identificando la relación de adelanto o retraso entre las mismas. 

\shortsolution{
Respuesta: $i_x(t)$ retrasa $22.83^{\circ{}}$ a $i_L(t)$ o $i_L(t)$ adelanta $22.83^{\circ{}}$ a $i_x(t)$.
}

\solution{
\textbf{Solución:}

Tomando el circuito en el dominio fasorial:

\begin{center}
  \includegraphics[scale=1]{prob02_sol}\calif{1\,pto}
\end{center}

Analizando los nodos 
 
\setlength{\columnsep}{20pt}      % Espacio entre columnas
\setlength{\columnseprule}{0.4pt} % Grosor de la línea divisoria 

\begin{multicols}{2}

\begin{align*}
\mathbf{I}_L &= \dfrac{(4\angle{15^{\circ{}}})(-4j)}{-4j+12j}\\
\mathbf{I}_L &= 2\angle{-165^{\circ{}}}\rightarrow\calif{1\,pto}
\end{align*}

\begin{align*}
\mathbf{I}_y &= \dfrac{(4\angle{15^{\circ{}}})(8)}{8+6j}\\
\mathbf{I}_y &= \dfrac{16}{5}\angle{-21.87^{\circ{}}}\rightarrow\calif{1\,pto}
\end{align*}
\end{multicols}

Luego, se analiza el nodo común de los inductores, donde se debe cumplir:

\begin{align*}
\mathbf{I}_x + \mathbf{I}_y &= \mathbf{I}_L\\
\mathbf{I}_x &= \mathbf{I}_L - \mathbf{I}_y\\
\mathbf{I}_x &= 2\angle{-165^{\circ{}}} - \dfrac{16}{5}\angle{-21.87^{\circ{}}}\\
\mathbf{I}_x &= 4.96\angle{172.16^{\circ{}}}\rightarrow\calif{2\,ptos}\\
\end{align*}

Retomando el análisis final, el desfase planteado se concreta como:

\begin{align*}
\theta_x-\theta_L &= \theta_{desfase}\\
172.16^{\circ{}} - -165^{\circ{}} &= \theta_{desfase}\\
337,17^{\circ} &= \theta_{desfase}\\
337,17^{\circ}-360^{\circ{}} &= \theta_{desfase}\\
\theta_{desfase} &= -22.83^{\circ{}}\rightarrow\calif{1\,pto}
\end{align*} 

Por lo tanto, $i_x(t)$ retrasa $22.83^{\circ{}}$ a $i_L(t)$ o $i_L(t)$ adelanta $22.83^{\circ{}}$ a $i_x(t)$. \calif{1\,pto}
 
}