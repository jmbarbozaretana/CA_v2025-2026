%%%%%%%%%%%%%%%
%% Problema1 %%
%% 2025 S2   %%
%% SGB       %%

\Problema{10}{Análisis de Circuitos en CA}

\medskip

Para el circuito mostrado, la fuente de corriente tiene la forma de onda $i(t)$.

\begin{center}
  \includegraphics[scale=1]{prob01a} \\
  \vspace{2ex}
  \includegraphics[scale=1]{prob01b}
\end{center}

\begin{subpunto}
\item Analice la forma de onda de $i(t)$ y determine su amplitud, periodo y frecuencia angular. A partir de lo cual determine el fasor $\phr{I}$ asociado a $i(t)$. Además calcule analíticamente $\phr{I}_1$, $\phr{I}_2$ y $\phr{V}$. \partialPoints{5}

\shortsolution{\textbf{Respuesta:} 
$I_m = 4 \, A$, $T = 16 \, ms$, $\omega = 125\pi \approx 392.7 \, rad/s$, $\phr{I} = 4 \angle 45^\circ$, $\phr{I}_1 = 3.58 \angle 18.43^\circ$, $\phr{I}_2 = 1.79 \angle 108^\circ$ y $\phr{V} = 7.16 \angle 18.43^\circ$.

}

\solution{

\textbf{Solución:}

De la lectura de la gráfica se lee fácilmente que $I_m = 4 \, A$ y que $T = 16 \, ms$. Esto permite calcular
\begin{equation*}
  \omega = \dfrac{2 \pi}{T} = \dfrac{2 \pi}{16 \times 10^{-3}} = 125\pi \approx 392.7 \, rad/s.
\end{equation*}

El desfase se lee de la forma de onda como $2 \, ms$ de adelanto y, por regla de tres:
\begin{equation*}
  \theta_i = \dfrac{360^\circ (2 \, ms)}{16 \, ms} = 45^\circ 
\end{equation*}

Así el fasor de corriente de la fuente sería
\begin{equation*}
  \phr{I} = 4 \angle 45^\circ
\end{equation*}

\calif{2 ptos hasta esta respuesta.}

El voltaje se puede calcular con la impedancia equivalente:
\begin{align*}
  \phr{V} &= \phr{I} \phr{Z}_{eq} = (\phr{I} \angle 45^\circ) \left( \dfrac{-j6}{2 - j4} \right) \\
          &= 7.155 \angle 18.43^\circ \calif{1 ptos.}
\end{align*}

y las corrientes por ley de Ohm:
\begin{align*}
  \phr{I}_1 &= \dfrac{7.155 \angle 18.43^\circ}{2} 
             = 3.577 \angle 18.43^\circ \calif{1 ptos.} \\
  \phr{I}_2 &= \dfrac{7.155 \angle 18.43^\circ}{-j4} 
             = 1.7883.577 \angle 108.43^\circ \calif{1 ptos.}
\end{align*}

}

\item Utilizando una escala de $1 \, cm = 1 \, A$ ó $1 \, cm = 1 \, V$, según sea necesario, dibuje el diagrama de fasores para $\phr{I}$, $\phr{I}_1$, $\phr{I}_2$ y $\phr{V}$. Etiquete claramente el gráfico con magnitudes y ángulos. \partialPoints{4}

\shortsolution{\textbf{Respuesta:} 

\begin{center}
  \includegraphics[scale=1.5]{prob01a_sol}
\end{center}

}

\solution{

\textbf{Solución:}

\begin{center}
  \includegraphics[scale=1.5]{prob01a_sol}
\end{center}

}

\item ¿Cómo se cumple gráficamente la LCK en el diagrama de fasores? \partialPoints{1}

\shortsolution{\textbf{Respuesta:} 

Gráficamente se cumple que $\phr{I} = \phr{I}_1 + \phr{I}_2$.
\begin{center}
  \includegraphics[scale=1.5]{prob01b_sol}
\end{center}

}

\solution{

\textbf{Solución:}

Gráficamente se cumple que $\phr{I} = \phr{I}_1 + \phr{I}_2$.
\begin{center}
  \includegraphics[scale=1.5]{prob01b_sol}
\end{center}

}

\end{subpunto}
