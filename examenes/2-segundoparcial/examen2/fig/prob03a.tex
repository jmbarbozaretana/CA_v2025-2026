\documentclass[class=minimal,border=2pt]{standalone}
\usepackage{tikz}
\usepackage{verbatim}
\usepackage[siunitx,americaninductors,smartlabels]{circuitikz}

\begin{document}
\begin{circuitikz}[american voltages, american currents, american resistors]
% --- Barra de tierra común ---
\draw (1,-3) node[ground]{} -- (10,-3);  % referencia inferior

% --- Fuente Vs (positivo arriba) → R1 → v_a, con C a tierra ---
\draw (1,-3) to[V, l=$v_s$, invert, *-] (1,0)   % positivo arriba
      to[R, l=$R_1$] (4,0) coordinate(va)
      (va) to[C, l_= $C$, *-*] (4,-3);

% --- OpAmp (– arriba, + abajo) ---
\draw (6,0.5) node[op amp] (U1) {};
\draw (va) -- (U1.+);  % entrada no inversora (inferior)

% --- Red R2–R3 (divisor y realimentación) ---
\coordinate (n) at (4,2);
\draw (1,2) to[R, l=$R_2$, -*] (n);
\draw (1,1.5) node[ground]{} -- (1,2);  % R2 a tierra
\draw (n) to[R, l=$R_3$] (7.5,2) -- (7.5,0.5);    % R3 hasta salida
\draw (n) -- (4,1) -- (U1.-);         % n a entrada inversora

% --- Salida: v_b, R4, C de salida ---
\draw (U1.out) -- (7.5,0.5) coordinate(vb)
      to[R, l=$R_4$, *-] (10,0.5) coordinate(vo)
      (vo) to[C, l_= $C$] (10,-3);

% --- Voltajes en los nodos ---
\draw
($(va)+(0.75,-0.25)$) to[open, v=$v_a$] ($(va)+(0.75,-2.75)$);
\draw
($(vb)+(0,0)$) to[open, v=$v_b$] ($(vb)+(0,-3.35)$);
\draw
($(vo)+(0.75,0)$) to[open, v=$v_o$] ($(vo)+(0.75,-3.35)$);


\end{circuitikz}
\end{document}
