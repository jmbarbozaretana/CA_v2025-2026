%% 2do Examen parcial S2 2025
%% Laura Cabrera Quirós

\Problema{11}{Respuesta en Frecuencia y Filtros}
\vspace{2mm}

Considere el circuito de la Figura \ref{fig:filtro}, que corresponde a un filtro pasa banda donde se sabe que su frecuencia central está definida por $\dfrac{1}{RC}$.

\begin{figure}[h]
  \centering
  \includegraphics[scale=1]{prob02.pdf}
  \caption{Circuito filtro pasa banda}
  \label{fig:filtro}
\end{figure}

Resuelva lo siguiente:

\begin{subpunto}

  \item Transforme el circuito a su equivalente fasorial, denotando el valor de todos sus componentes (en términos literales). \partialPoints{1}
  
  \solution{
  	\begin{center}
  \includegraphics[scale=1]{prob02_sol.pdf}
  	\end{center}
  \calif{Debe venir en fasores con $\omega$ como variable, usar solo $C$ en lugar de $Z_c$ hace que se pierdan todos los puntos.}
  }
  
  \item Encuentre la función de respuesta en frecuencia  $\phr{H}(\omega)$ de este circuito como función de $C$ y $R$, tomando $\phr{V}_s$ como entrada y $\phr{V}_o$ como salida. Debe desarrollar su respuesta hasta llegar a una división de polinomios. \partialPoints{5}
  
  \solution{
  	Por divisor de tensión:
  \begin{align*}
    \phr{V}_o &= \dfrac{\phr{V}_aR}{R+1/j\omega C}\\
               &= \dfrac{\phr{V}_aj\omega RC}{j\omega RC+1} \\
    \phr{V}_a &= \dfrac{\phr{V}_o(j\omega RC+1)}{j\omega RC}  \calif{1Punto}
  \end{align*}
  
  Además, mediante un nodo y sustituyendo expresión anterior:
  
  \begin{align*}
  	\dfrac{\phr{V}_s-\phr{V}_a}{R}&=\dfrac{\phr{V}_a}{1/j\omega C}+\dfrac{\phr{V}_a}{1/j\omega C+1}\\
  	\phr{V}_s&=\phr{V}_a\left(1+j\omega RC+\dfrac{j\omega RC}{j\omega RC+1}\right) \calif{1Puntos}\\
  	\phr{V}_s&=\phr{V}_o\dfrac{(j\omega RC+1)}{j\omega RC}  \left(1+j\omega RC+\dfrac{j\omega RC}{j\omega RC+1}\right) \\
  	\phr{V}_s&=\phr{V}_o\left(\dfrac{j\omega RC+1}{j\omega RC}+(j\omega RC+1)+1\right)\\
  	\phr{V}_s&=\phr{V}_o\left(\dfrac{j\omega RC+1}{j\omega RC}+j\omega RC+2\right)\calif{1 Punto}
  \end{align*}
  
  Y finalmente, desarrollando para obtener $\phr{H}(\omega)$ como una división de polinomios:
  
  \begin{align*}
  	\phr{V}_s&=\phr{V}_o\left(\dfrac{3j\omega RC-(\omega RC)^2+1}{j\omega RC}\right)\\
  	\phr{H}(\omega)&=\dfrac{\phr{V}_o}{\phr{V}_s}=\dfrac{j\omega RC}{3j\omega RC+1-(\omega RC)^2} \calif{2 Puntos}
  \end{align*}
  
  \calif{Tener un equivalente de la función correcta, pero sin desarrollo a polinomios quitaría un punto}
}    
  \item Encuentre la respuesta en magnitud para este filtro. \partialPoints{1}
  
  \solution{  
  	 \begin{align*}
  		|\phr{H}(\omega)| &= \left|\dfrac{j\omega RC}{3j\omega RC+1-(\omega RC)^2}\right|=\dfrac{\omega RC}{\sqrt{(3\omega RC)^2+(1-(\omega RC)^2)^2}}\calif{1 Punto}\\
  	\end{align*}
  }
  
  \item Calcule el valor máximo de ganacia en decibeles para este filtro. \partialPoints{1}


\solution{  
  

  En el enunciado se da la frecuencia de resonancia, en donde teóricamente se pueden obtener un máximo para la ganancia en un pasabanda. Evaluando en este punto:
  \begin{align*}
    |\phr{H}(1/RC)| &=\dfrac{(1/RC) RC}{\sqrt{(3(1/RC) RC)^2+(1-((1/RC) RC)^2)^2}}=\dfrac{1}{3}\calif{0.5 Punto}\\
    |\phr{H}(1/RC)|_{dB} &=20log_{10}(1/3)-9.54dB\calif{0.5 Punto}
  \end{align*}

  }
    
  \item Si se definen los valores de $C=1m$F y $R=1k\Omega$, encuentre las frecuencias de corte y el ancho de banda para el filtro, todas en $rad/s$.\partialPoints{3}
  
  \solution{

 Para los valores dados:
   \begin{align*}
 	\phr{H}(\omega)&=\dfrac{j\omega}{3j\omega+1-(\omega)^2} 
 \end{align*}
 Y la respuesta en magnitud:
  \begin{align*}
 	|\phr{H}(\omega)| &= \dfrac{\omega}{\sqrt{(3\omega)^2+(1-\omega^2)^2}}=\dfrac{\omega}{\sqrt{9\omega^2+(1-2\omega^2+\omega^4)}}=\dfrac{\omega}{\sqrt{\omega^4+7\omega^2+1}}\\
 \end{align*}
  Finalmente, las frecuencias de corte se dan en el 70\% del valor maximo. Asi, tomando la respuesta del punto anterior:
  
	\begin{align*}
	   	\dfrac{\omega}{\sqrt{\omega^4+7\omega^2+1}} &= \dfrac{|\phr{H}(\omega)|_{max}}{\sqrt{2}}\\
	   	\dfrac{\omega}{\sqrt{\omega^4+7\omega^2+1}}&= \dfrac{1}{3\sqrt{2}}\\
	   	\dfrac{\omega^2}{\omega^4+7\omega^2+1}&= \dfrac{1}{9(2)}\\
		\omega^4+7\omega^2+1&=18\omega^2\\
	   	\omega^4-11\omega^2+1&=0\
	\end{align*}
	Lo que genera cuatro soluciones, de las cuales las validas para el circuito serian:
	\begin{align*}
		\omega_1&=0.3 rad/s\\
		\omega_2&=3.3 rad/s
	\end{align*}
	Lo que genera un ancho de banda de $\sim3rad/s$
  }
    
  
  \end{subpunto}
    

    
