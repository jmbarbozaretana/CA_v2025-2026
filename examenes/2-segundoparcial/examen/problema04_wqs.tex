%% 2do Examen parcial S2 2025
%% William Quirós Solano

\Problema{10}{Análisis de Circuitos con Transformada de Laplace}
\vspace{2mm}

Para el circuito de la Figura \ref{fig:laplace} donde el interruptor $t_0$ se accciona en $t=0\,seg$ y los elementos reactivos inicialmente no tienen energía almacenada.

\begin{figure}[ht]
  \centering
  \includegraphics[scale=1.2]{prob04}
  \caption{Circuito para análisis con Transformada de Laplace}
  \label{fig:laplace}
\end{figure}
    
Responda lo siguiente:    
\begin{subpunto}
\item Dibuje el circuito en el dominio $s$. \partialPoints{1}

\solution{
\begin{center}
\includegraphics[scale=1]{prob04_sol_a}
\end{center}
}

\item Obtenga una expresión para la corriente en el condensador $I_C$ en el dominio $s$. Simplifique al máximo. \partialPoints{4}.

\solution{
Aplicando LCK en el nodo de la bobina
\begin{align*}
\dfrac{\frac{480}{s}-V_L}{20}-\dfrac{V_L}{0.002s}-\dfrac{V_L}{60+\frac{2\times 10^5}{s}}=0
\end{align*}
\calif{1 pt por la ecuación de nodo correcta}

Sabiendo que la corriente en el condensador es
\begin{align*}
I_C=\dfrac{V_L}{60+\frac{2\times 10^5}{s}}=\dfrac{sV_L}{60s+2\times 10^5}
\end{align*}
\calif{1 pt por la ecuación para la corriente en el condensador}

despejando de la ecuación de nodo
\begin{align*}
\dfrac{480-sV_L}{20s}-\dfrac{500V_L}{s}-\dfrac{sV_L}{60s+2\times 10^5}=0\\
V_L\Big(\dfrac{1}{20}+\dfrac{500}{s}+\dfrac{s}{60s+2\times 10^5}\Big)=\dfrac{24}{s}\\
V_L\Big(\dfrac{s+1\times 10^4}{20s}+\dfrac{s}{60s+2\times 10^5}\Big)=\dfrac{24}{s}\\
V_L\Big(\dfrac{60s^2+2s\times 10^5+60s\times 10^4+2\times10^9 +20s^2}{20s(60s+2\times10^5)}\Big)=\dfrac{24}{s}\\
V_L=\dfrac{480(60s+2\times 10^5)}{80s^2+8\times10^5+2\times10^9}\\
V_L=\dfrac{6(60s+2\times 10^5)}{(s^2+10000s+25\times10^6)}
\end{align*}
\calif{1 pt por la ecuación de la tensión de nodo}

por lo que la corriente en el condensador es
\begin{align*}
I_C=\dfrac{6(60s+2\times 10^5)}{(s^2+10000s+25\times10^6)}\dfrac{s}{(60s+s\times10^5)}=\dfrac{6s}{(s^2+10000s+25\times10^6)}\\
I_C=\dfrac{6s}{(s+5000)^2}
\end{align*}
\calif{1 pt por la ecuación final}
}

\item Obtenga la expresión para la corriente en el condensador $i_C(t)$ en el dominio temporal. Simplifique al máximo. \partialPoints{2}.

\solution{
Se procede con las fracciones parciales dado que la expresión es propia
\begin{align*}
I_C=\dfrac{6s}{(s+5000)^2}=\dfrac{A}{(s+5000)}+\dfrac{B}{(s+5000)^2}
\end{align*}
donde las constantes son
\begin{align*}
B=\lim_{s\rightarrow -5000} (s+5000)^2 \dfrac{6s}{(s+5000)^2}= -30000\\
A=\lim_{s\rightarrow -5000} \dfrac{d}{ds}\Big[(s+5000)^2 \dfrac{6s}{(s+5000)^2}\Big]=6
\end{align*}
\calif{0.5 pts por cada constante}

con esto entonces
\begin{align*}
I_C=\dfrac{6}{(s+5000)}-\dfrac{30000}{(s+5000)^2}
\end{align*}
y obteniendo la transformada inversa
\begin{align*}
i_C(t)=6e^{-5000t}u(t)-30000te^{-5000t}u(t)\,A\\
i_C(t)=(6-30000t)e^{-5000t}u(t)\,A
\end{align*}
\calif{1 pt por el cálculo de la expresión final}
}

\item Esboce la gráfica de $i_C(t)$. Utilice como escala horizontal $1\,cm=25\,\mu s$ y escala vertical $1\,cm=1\,A$.\partialPoints{3}.

\solution{
La gráfica para la corriente en el condensador es
\begin{center}
\includegraphics[scale=1]{prob04_sol_4.png}
\end{center}
Con la ecuación y la escala propuesta se pueden considerar los siguientes valores para la revisión
\begin{center}
\begin{tabular}{c|c}
\hline
Tiempo [$\mu s$] & Corriente [A]\\
\hline
0 & \textbf{6}\\
\hline
25 & 4.63\\
\hline
50 & 3.50\\
\hline
75 & 2.57\\
\hline
100 & 1.81\\
\hline
125 & 1.20\\
\hline
150 & 0.70\\
\hline
175 & 0.31\\
\hline
200 & \textbf{0}\\
\hline
\end{tabular}
\end{center}
\calif{1 punto por cada extremo de la gráfica o algún punto intermedio y 1 punto por la tendencia correcta de la gráfica}
}
\end{subpunto}