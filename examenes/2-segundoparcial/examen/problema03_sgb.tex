%% 2do Examen parcial S2 2025
%% Saúl Guadamuz Brenes


\Problema{10}{Respuesta en Frecuencia y diagramas de Bode}

\vspace{2mm}

%Considere el circuito de la figura \ref{fig:cir_bode}, cuya entrada es el voltaje $v_s$ de la fuente y la salida es el voltaje $v_o$ en el condensador.

%\begin{figure}[htp]
%  \centering
%  \includegraphics[scale=1]{prob03a}
%  \caption{Circuito para análisis en frecuencia.} 
%  \label{fig:cir_bode}
%\end{figure}

Se sabe que un circuito tiene la respuesta en magnitud que se representa en el diagrama de Bode de la Figura \ref{fig:diag_bode}.



\begin{figure}[htp]
  \centering
  \includegraphics[scale=1.25]{prob03b}
  \caption{Respuesta en frecuencia.}
  \label{fig:diag_bode}
\end{figure}

\begin{subpunto}
	
	
%\item Determine la función de respuesta en frecuencia $\mathbf{H}(\omega) = \mathbf{V}_o(\omega) / \mathbf{V}_s(\omega)$ del circuito de la Figura \ref{fig:cir_bode}. Exprésela en la forma estándar. \partialPoints{3}
%
%\solution{
%
%Se podría realizar el análisis de la forma tradicional (plantenado nodos desde la entrada) pero es más fácil observar que se trata de un amplificador no inversor cuya entrada es $v_a$ y su salida es $v_b$. $v_a$ se puede calcular con un divisor de tensión a partir de $v_s$, mientras que $v_o$ se puede calcular con otro divisor de tensión a partir de $v_b$; claramente todo debe estar en el dominio fasorial.
%
%Para el cálculo de $\phr{V}_a$:
%\begin{align*}
%	\phr{V}_a &= \dfrac{\phr{V}_s (1/j \omega C)}{R_1 + 1/j \omega C}
%			  = \dfrac{\phr{V}_s (1/j \omega C)}{(1/j \omega C) (1 + j \omega R_1 C)} \\
%			  &= \dfrac{\phr{V}_s}{1 + j \omega R_1 C} \quad \calif{0.5\,pt}
%\end{align*}
%
%
%Como se trata de un amplificador no inversor:
%\begin{align*}
%	\phr{V}_b &= \phr{V}_a (1 + R_3/R_2) \\
%			  &= \dfrac{\phr{V}_s (1 + R_3/R_2)}{1 + j \omega R_1 C} \quad \calif{1\,pt}
%\end{align*}
%
%La salida nuevamente con un divisor de tensión:
%\begin{align*}
%	\phr{V}_o &= \dfrac{\phr{V}_b (1/j \omega C)}{R_4 + 1/j \omega C}
%			  = \dfrac{\phr{V}_b (1/j \omega C)}{(1/j \omega C) (1 + j \omega R_4 C)} \\
%			  &= \dfrac{\phr{V}_b}{1 + j \omega R_4 C} \\
%			  &= \dfrac{\phr{V}_s (1 + R_3/R_2)}{(1 + j \omega R_1 C)(1 + j \omega R_4 C)}   \quad \calif{0.5\,pt}
%\end{align*}
%
%Con lo que la respuesta en frecuencia sería:
%\begin{equation*}
%	\phr{H}(\omega) = \dfrac{\phr{V}_o}{\phr{V}_s} = \boxed{\dfrac{(1 + R_3/R_2)}{(1 + j \omega R_1 C)(1 + j \omega R_4 C)}} \quad \calif{1\,pt}
%\end{equation*}
%
%}
    
\item Obtenga la expresión de la función de respuesta en frecuencia de la Figura \ref{fig:diag_bode}. Exprésela en la forma estándar. Nótese que el valor del gráfico en $\omega = 0.1 \, rad/s$ es de $32 \, dB$  los quiebres en magnitud ocurren en $8 \, rad/s$ y $320 \, rad/s$. El ángulo del término de ganancia es de $0^\circ$. \partialPoints{3}
    
\solution{

Examinando el gráfico, se encuentra:
\begin{itemize}
	\item Término de ganancia: $10^{32/20} = 40$.
	\item Un polo simple en $8 \, rad/s$.
	\item Un polo simple en $320 \, rad/s$.
\end{itemize}

combinando estos elementos, la función de respuesta en frecuencia sería:
\begin{equation*}
	\phr{H}(\omega) = \boxed{\dfrac{40}{(1 + j \omega/8)(1 + j \omega/320)}}
\end{equation*}

\calif{1\,pt cada término}
}

    
\item Si la función de transferencia que da origen a la respuesta de la Figura \ref{fig:diag_bode} está dada por:
\begin{equation*}
	\phr{H}(\omega) = \dfrac{\phr{V}_o}{\phr{V}_s} = \dfrac{(1 + R_3/R_2)}{(1 + j \omega R_1 C)(1 + j \omega R_4 C)}
\end{equation*}
donde $R_1$, $R_2$, $R_3$ y $R_4$ corresponden a resistencias y $C$ es una capacitancia y se tiene que $C = 0.5 \, \mu F$ y $R_2 = 10 \, k\Omega$. 

Calcule los valores de $R_1$, $R_3$ y $R_4$ para que esta función de transferencia dada tenga la respuesta en frecuencia mostrada en la Figura \ref{fig:diag_bode}. \partialPoints{3}
    
\solution{

Igualando las respuestas en frecuencia de las dos partes anteriores
\begin{equation*}
	\dfrac{40}{(1 + j \omega/8)(1 + j \omega/320)} = \dfrac{(1 + R_3/R_2)}{(1 + j \omega R_1 C)(1 + j \omega R_4 C)}  
\end{equation*}
e igualando términos se obtienen las igualdades:
\begin{align*}
	1 + \dfrac{R_3}{R_2} &= 40 \\
	R_1 C &= \dfrac{1}{8} \\
	R_4 C &= \dfrac{1}{320}
\end{align*}

Con los valores de $C$ y $R_2$ dados, el resto se calcula como:
\begin{align*}
	R_1 &= \dfrac{1}{8(0.5 \, \mu F)} = \boxed{250 \, k \Omega} \quad \calif{1\,pt} \\
	R_3 &= (40-1)10 \, k \Omega = \boxed{390 \, k \Omega}  \quad \calif{1\,pt} \\
	R_4 &= \dfrac{1}{320(0.5 \, \mu F)} = \boxed{6.25 \, k \Omega}  \quad \calif{1\,pt}
\end{align*}



}

\item ¿Son los valores de $C$, $R_1$, $R_2$, $R_3$ y $R_4$ encontrados en el ítem anterior los únicos posibles para lograr la respuesta en frecuencia solicitada? Justifique su respuesta. \partialPoints{1}

\solution{

La solución no es única. Esto se puede justificar ya sea con el concepto de escalamiento en magnitud o viendo que, al escoger otros valores de $C$ y $R_2$, se calculan otros valores de $R_1$, $R_3$ y $R_4$. \calif{1\,pt}
}

\item Si a la función de transferencia de la Figura \ref{fig:diag_bode} se le agregara un cero en el origen, ¿cómo cambiaría el diagrama de Bode de magnitud? Realice el diagrama del mismo en hoja semilogarítmica. \partialPoints{3}
    
\solution{

Al agregar un cero en el origen, se agrega una asíntota adicional:
\begin{figure}[htp]
	\centering
		\includegraphics[scale=1]{prob03_sol1}
	\caption{ }
	\label{fig:cir_bode1}
\end{figure}

que cambia el inicio del diagrama a una pendiente de $20 \, dB/dec$ desde $12 \, dB$ en $\omega = 0.1 \, rad/s$ hasta $\omega = 8 \, rad/s$, a continuación la pendiente queda en $0 \, dB/dec$ hasta $\omega = 320 \, rad/s$ desde donde baja a $-20 \, dB/dec$, formando así un comportamiento pasa-banda.
\begin{figure}[htp]
	\centering
		\includegraphics[scale=1]{prob03_sol2}
	\caption{ }
	\label{fig:cir_bode2}
\end{figure}


}
 
\end{subpunto}





