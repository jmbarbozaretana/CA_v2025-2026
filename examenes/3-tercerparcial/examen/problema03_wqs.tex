%% 3er Examen parcial S2 2025
%% William Quirós

\Problema{8}{Redes de dos puertos}

\vspace{2mm}

Dado el curcuito de la siguiente figura:

\begin{center}
\includegraphics[scale=1]{prob03_wqs}
\end{center}

Determine el valor de los cuatro paramétros $h$ del circuito anterior.

\solution{
El parámetro $h_{11}$ se determina con
\begin{align*}
\mathbf{h}_{11} = \dfrac{\mathbf{V}_1}{\mathbf{I}_1}\Bigg|_{\mathbf{V}_2=0}
\end{align*}
aplicando LTK en el puerto de entrada
\begin{align*}
V_1-400I_1-1200I_1=0\\
V_1=1600I_1\\
\dfrac{V_1}{I_1}=1600
\end{align*}
por lo tanto
\begin{align*}
h_{11}=1600\,\Omega
\end{align*}
\calif{2 pts}

El parametro $h_{12}$ se detemina con
\begin{align*}
\mathbf{h}_{12} = \dfrac{\mathbf{V}_1}{\mathbf{V}_2}\Bigg|_{\mathbf{I}_1=0}
\end{align*}
se observa que cuando el puerto de entrada se abre
\begin{align*}
V_1-400I_1-1200I_1=0\\
V_1=0
\end{align*}
y como $V_1$ también es la tensión de ambas terminales del aplificador
\begin{align*}
\dfrac{0}{V_2}=0
\end{align*} 
por lo que 
\begin{align*}
h_{12}=0
\end{align*}
\calif{2 pts}

El parametro $h_{21}$ se detemina con
\begin{align*}
\mathbf{h}_{21} = \dfrac{\mathbf{I}_2}{\mathbf{I}_1}\Bigg|_{\mathbf{V}_2=0}
\end{align*}
se observa que $V_2=0$ y que la corriente de salida se puede escribir como
\begin{align*}
I_2=\dfrac{0-V_{sal}}{200}
\end{align*}
y como
\begin{align*}
\dfrac{V_{sal}500}{1000+500}=\dfrac{V_1 1200}{1200+400}\\
\dfrac{V_{sal}}{3}=\dfrac{3V_1}{4}\\
\dfrac{V_{sal}}{3}=\dfrac{3(400+1200I_1)}{4}\\
V_{sal}=3600I_1
\end{align*}
por lo que 
\begin{align*}
I_2=-\dfrac{V_{sal}}{200}=-\dfrac{3600I_1}{200}\\
\dfrac{I_2}{I_1}=-18
\end{align*}
por tanto
\begin{align*}
h_{21}=-18
\end{align*}
\calif{2 pts}

El parametro $h_{22}$ se detemina con
\begin{align*}
\mathbf{h}_{22} = \dfrac{\mathbf{I}_2}{\mathbf{V}_2}\Bigg|_{\mathbf{I}_1=0}
\end{align*}
se observa que $V_{sal}=0$ dado que las tensiones en las terminales inversora y no inversora es la misma, de manera que
\begin{align*}
\dfrac{V_2-V_{sal}}{200}=I_2\\
\dfrac{V_2}{200}=I_2\\
\dfrac{I_2}{V2}=\dfrac{1}{200}
\end{align*}
por lo que
\begin{align*}
h_{22}=\dfrac{1}{200}\,S=0.005\,S=5\,mS
\end{align*}
\calif{2 pts}

}

