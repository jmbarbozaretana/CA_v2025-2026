% Encabezado

\rule{\textwidth}{1.5pt}

\begin{minipage}[t]{110mm}
  \begin{tabular}{l}
    Instituto Tecnológico de Costa Rica \\
    Escuela de Ingeniería Electrónica \\
    EL-2114 Circuitos Eléctricos en Corriente Alterna \\
    \begin{tabular}{@{}ll}
    Profesores: &M.Sc. José Miguel Barboza Retana \\
                &Dr. William Quirós Solano\\                 
                &Dra. Laura Cabrera Quirós \\
                &Dr. Saúl Guadamuz Brenes
    \end{tabular}\\
    II Semestre 2025 \\
    Tercer Examen Parcial\\
    Miércoles 03 de diciembre 2025\\
    Hora de inicio: 1:00 pm\\
    Duración: 3 horas
  \end{tabular}
\end{minipage}
\hfill
\begin{minipage}[t]{4cm}
  \begin{tabular}{|l|l|}
    \hline
    Total de Puntos: & 38 \\[2mm]
    \hline
    Puntos obtenidos: & \rule{10mm}{0pt}\\[2mm]
    \hline
    Porcentaje: & \\[2mm]
    \hline
    Nota: & \\[2mm]
    \hline
  \end{tabular}
\end{minipage}
\hspace{12mm}

\rule{\textwidth}{1.5pt}

\medskip

Nombre del estudiante: \linea{0.4\textwidth} \hfill Carné: \linea{0.2\textwidth}

\vspace{5mm}

\fbox{\parbox{\textwidth}{\textbf{Instrucciones Generales:} 
    %
    \begin{itemize}
    \item Resuelva el examen en forma clara, ordenada e individual.
    \item Se permite el uso del formulario oficial del curso, mismo que ha sido dispuesto por el profesor a cargo. Este documento solo puede utilizarse en modo impreso y sin anotaciones adicionales sobre el mismo.
    \item No se aceptarán reclamos de desarrollos con lápiz, borrones o 
      corrector de lapicero.
    \item Si trabaja con lápiz, debe encerrar en recuadro su respuesta final con lapicero.
    \item El uso de lapicero rojo no está permitido.
    \item El uso del teléfono celular no es permitido.  Este tipo de 
      dispositivos debe permanecer totalmente apagado durante
      el examen.
    \item No se permite el uso de calculadora programable.
    \item Durante el desarrollo del examen solo se atienden dudas de forma.
    \item El instructivo del examen debe ser devuelto junto con su solución.
    \item El no cumplimiento de los puntos anteriores equivale a una nota igual a cero en el ejercicio correspondiente o en el examen.
     \item Esta prueba tiene una duración de 3 horas a partir de su hora de inicio.
     \medskip
      \begin{center}
        \textbf{Firma del estudiante:} \rule{7cm}{0.4pt} 
    \end{center}
    \end{itemize}}}
   
\clearpage
 
\textcolor{white}{H} 
\vspace{5cm}
 
\begin{center}   
	\Large{\textbf{Detalle del Contenido del Examen}}
\end{center}

\normalsize
\bigskip    

\begin{center}
  \begin{tabular}{|>{\centering\arraybackslash}m{2.5cm}|>{\centering\arraybackslash}m{2.5cm}|>{\centering\arraybackslash}m{2.5cm}|}
    \hline
    \textbf{Ítem} & \textbf{Puntos totales} &  \textbf{Puntos obtenidos} \\[2mm]
    \hline
    Problema 1 & 8 &  \\[2mm]
    \hline
    Problema 2 & 12 &  \\[2mm]
    \hline
    Problema 3 & 8 &  \\[2mm]
    \hline
    Problema 4 & 10 &  \\[2mm]
    \hline
  \end{tabular}
\end{center}

\clearpage

%\renewcommand{\EstimatedTimePerPoint}{562} %% 562 por punto
%\renewcommand{\EstimatedTimeLabel}{} %% Originalmente es TE:

%\section*{Preguntas Cortas \TotalPreguntas{6}}


%%% 2do Examen parcial S2 2021
%% José Miguel Barboza Retana 

\Pregunta{2}{Considere el siguiente sistema trifásico de 4 hilos el cual es alimentado por una fuente balanceada con secuencia negativa y tensión de línea de $\mathbf{V_{ab}}=120\angle{20^{\circ{}}}\,V_{rms}$:}

\begin{center}
  \includegraphics[scale=1]{p02}
\end{center}
    
Además, considerando que las impedancias correspondientes de la carga son $\mathbf{Z_{1}}=40\,\Omega$, $\mathbf{Z_{2}}=50\,\Omega$ y $\mathbf{Z_{3}}=30+20j\,\Omega$, ¿cuál de las siguientes opciones corresponde a la lectura correcta del vatímetro $W_{2}$?

\abcdE 
{$288.33\,$W\medskip}
{$181.02\,$W\medskip}
{$311.49\,$W\medskip}
{$92.28\,$W\medskip}
{$110.77\,$W\medskip}
%\problem{\vspace{5cm}\vfill}

%\vfill
%\problem{\clearpage}

\section*{Desarrollo}

%% 3er Examen parcial S2 2025
%% José Miguel Barboza

\Problema{8}{Series de Fourier}

\vspace{2mm}

Sea la señal $x(t)$ que se muestra en la siguiente figura:

\begin{center}
\includegraphics[scale=1]{prob01}
\end{center}

Determine la expresión para la función de los coeficientes necesarios para una serie trigonométrica de Fourier que sintetice la función $x(t)$.

\solution{

\begin{itemize}
\item La función $x(t)$ presenta simetría par, por tanto $b_n=0$. \calif{1\,pto}
\item Al analizar el área bajo la curva de la función en un periodo, se aprecia que la misma es igual a 0, por lo tanto $a_0=0$. \calif{1\,pto}
\item Tomando el periodo de la señal como $T_0$, la frecuencia fundamental es igual a $\omega_0=\frac{2\pi}{T_0}$. \calif{1\,pto} 
\end{itemize}

Para el cálculo de los coeficientes $a_n$, lo primero es determinar la función matemática requerida para $x(t)$ en el intervalo a integrar. Al tener simetría para la señal $x(t)$, solo es requerido integrar un intervalo de tiempo de medio periodo iniciando en 0 y hasta $T_0/2$. En ese sentido, en dicho intervalo se tiene un segmento de una línea recta cuya ecuación $x(t)=mt+b$, se define para $m=\frac{-A-A}{T_0/2 -0}=-\frac{4A}{T_0}$ y $b=A$.

Así:

\begin{align*}
a_n &= \frac{4}{T_0}\int_{0}^{\frac{T_0}{2}}\left(-\dfrac{4A}{T_0}t+A\right)\cos\left(\dfrac{2\pi n}{T_0}t\right)\,dt\rightarrow\calif{1\,pto}\\
a_n &= \frac{4}{T_0}\dfrac{-4A}{T_0}\int_{0}^{\frac{T_0}{2}}t\cos\left(\dfrac{2\pi n}{T_0}t\right)\,dt + \frac{4}{T_0}A\int_{0}^{\frac{T_0}{2}}\cos\left(\dfrac{2\pi n}{T_0}t\right)\,dt\rightarrow\calif{1\,pto}\\
a_n &= -\dfrac{16A}{T_0^2}\dfrac{1}{\left(\dfrac{2\pi n}{T_0}\right)^2}\left[\cos\left(\dfrac{2\pi n}{T_0}t\right)+\dfrac{2\pi n}{T_0}t\sin\left(\dfrac{2\pi n}{T_0}t\right)\right]_{0}^{\frac{T_0}{2}}+\dfrac{4A}{T_0}\dfrac{\sin\left(\dfrac{2\pi n}{T_0}t\right)}{\dfrac{2\pi n}{T_0}}\bigg|_{0}^{\frac{T_0}{2}}\rightarrow\calif{1\,pto}\\
a_n &= -\dfrac{4A}{\pi^2n^2}\left[\cos(\pi n)+\pi n\sin(\pi n)-\cos(0)-0\sin(0)\right] + \dfrac{2A}{\pi n}\left[\sin(\pi n)-\sin(0)\right]\rightarrow\calif{1\,pto}\\
a_n &= -\dfrac{4A}{\pi^2n^2}\left(\cos(\pi n)-1\right)\rightarrow\calif{1\,pto}\\
a_n &= \dfrac{4A}{\pi^2n^2}\left(1-\cos(\pi n)\right)\\
\end{align*}


}

\problem{\vspace{5mm}}
%% 2do Examen parcial S2 2025
%% Laura Cabrera Quirós

\Problema{11}{Respuesta en Frecuencia y Filtros}
\vspace{2mm}

Considere el circuito de la Figura \ref{fig:filtro}, que corresponde a un filtro pasa banda donde se sabe que su frecuencia central está definida por $\dfrac{1}{RC}$.

\begin{figure}[h]
  \centering
  \includegraphics[scale=1]{prob02.pdf}
  \caption{Circuito filtro pasa banda}
  \label{fig:filtro}
\end{figure}

Resuelva lo siguiente:

\begin{subpunto}

  \item Transforme el circuito a su equivalente fasorial, denotando el valor de todos sus componentes (en términos literales). \partialPoints{1}
  
  \solution{
  	\begin{center}
  \includegraphics[scale=1]{prob02_sol.pdf}
  	\end{center}
  \calif{Debe venir en fasores con $\omega$ como variable, usar solo $C$ en lugar de $Z_c$ hace que se pierdan todos los puntos.}
  }
  
  \item Encuentre la función de respuesta en frecuencia  $\phr{H}(\omega)$ de este circuito como función de $C$ y $R$, tomando $\phr{V}_s$ como entrada y $\phr{V}_o$ como salida. Debe desarrollar su respuesta hasta llegar a una división de polinomios. \partialPoints{5}
  
  \solution{
  	Por divisor de tensión:
  \begin{align*}
    \phr{V}_o &= \dfrac{\phr{V}_aR}{R+1/j\omega C}\\
               &= \dfrac{\phr{V}_aj\omega RC}{j\omega RC+1} \\
    \phr{V}_a &= \dfrac{\phr{V}_o(j\omega RC+1)}{j\omega RC}  \calif{1Punto}
  \end{align*}
  
  Además, mediante un nodo y sustituyendo expresión anterior:
  
  \begin{align*}
  	\dfrac{\phr{V}_s-\phr{V}_a}{R}&=\dfrac{\phr{V}_a}{1/j\omega C}+\dfrac{\phr{V}_a}{1/j\omega C+1}\\
  	\phr{V}_s&=\phr{V}_a\left(1+j\omega RC+\dfrac{j\omega RC}{j\omega RC+1}\right) \calif{1Puntos}\\
  	\phr{V}_s&=\phr{V}_o\dfrac{(j\omega RC+1)}{j\omega RC}  \left(1+j\omega RC+\dfrac{j\omega RC}{j\omega RC+1}\right) \\
  	\phr{V}_s&=\phr{V}_o\left(\dfrac{j\omega RC+1}{j\omega RC}+(j\omega RC+1)+1\right)\\
  	\phr{V}_s&=\phr{V}_o\left(\dfrac{j\omega RC+1}{j\omega RC}+j\omega RC+2\right)\calif{1 Punto}
  \end{align*}
  
  Y finalmente, desarrollando para obtener $\phr{H}(\omega)$ como una división de polinomios:
  
  \begin{align*}
  	\phr{V}_s&=\phr{V}_o\left(\dfrac{3j\omega RC-(\omega RC)^2+1}{j\omega RC}\right)\\
  	\phr{H}(\omega)&=\dfrac{\phr{V}_o}{\phr{V}_s}=\dfrac{j\omega RC}{3j\omega RC+1-(\omega RC)^2} \calif{2 Puntos}
  \end{align*}
  
  \calif{Tener un equivalente de la función correcta, pero sin desarrollo a polinomios quitaría un punto}
}    
  \item Encuentre la respuesta en magnitud para este filtro. \partialPoints{1}
  
  \solution{  
  	 \begin{align*}
  		|\phr{H}(\omega)| &= \left|\dfrac{j\omega RC}{3j\omega RC+1-(\omega RC)^2}\right|=\dfrac{\omega RC}{\sqrt{(3\omega RC)^2+(1-(\omega RC)^2)^2}}\calif{1 Punto}\\
  	\end{align*}
  }
  
  \item Calcule el valor máximo de ganacia en decibeles para este filtro. \partialPoints{1}


\solution{  
  

  En el enunciado se da la frecuencia de resonancia, en donde teóricamente se pueden obtener un máximo para la ganancia en un pasabanda. Evaluando en este punto:
  \begin{align*}
    |\phr{H}(1/RC)| &=\dfrac{(1/RC) RC}{\sqrt{(3(1/RC) RC)^2+(1-((1/RC) RC)^2)^2}}=\dfrac{1}{3}\calif{0.5 Punto}\\
    |\phr{H}(1/RC)|_{dB} &=20log_{10}(1/3)-9.54dB\calif{0.5 Punto}
  \end{align*}

  }
    
  \item Si se definen los valores de $C=1m$F y $R=1k\Omega$, encuentre las frecuencias de corte y el ancho de banda para el filtro, todas en $rad/s$.\partialPoints{3}
  
  \solution{

 Para los valores dados:
   \begin{align*}
 	\phr{H}(\omega)&=\dfrac{j\omega}{3j\omega+1-(\omega)^2} 
 \end{align*}
 Y la respuesta en magnitud:
  \begin{align*}
 	|\phr{H}(\omega)| &= \dfrac{\omega}{\sqrt{(3\omega)^2+(1-\omega^2)^2}}=\dfrac{\omega}{\sqrt{9\omega^2+(1-2\omega^2+\omega^4)}}=\dfrac{\omega}{\sqrt{\omega^4+7\omega^2+1}}\\
 \end{align*}
  Finalmente, las frecuencias de corte se dan en el 70\% del valor maximo. Asi, tomando la respuesta del punto anterior:
  
	\begin{align*}
	   	\dfrac{\omega}{\sqrt{\omega^4+7\omega^2+1}} &= \dfrac{|\phr{H}(\omega)|_{max}}{\sqrt{2}}\\
	   	\dfrac{\omega}{\sqrt{\omega^4+7\omega^2+1}}&= \dfrac{1}{3\sqrt{2}}\\
	   	\dfrac{\omega^2}{\omega^4+7\omega^2+1}&= \dfrac{1}{9(2)}\\
		\omega^4+7\omega^2+1&=18\omega^2\\
	   	\omega^4-11\omega^2+1&=0\
	\end{align*}
	Lo que genera cuatro soluciones, de las cuales las validas para el circuito serian:
	\begin{align*}
		\omega_1&=0.3 rad/s\\
		\omega_2&=3.3 rad/s
	\end{align*}
	Lo que genera un ancho de banda de $\sim3rad/s$
  }
    
  
  \end{subpunto}
    

    

\problem{\vspace{5mm}}
%% 3er Examen parcial S2 2025
%% William Quirós

\Problema{8}{Redes de dos puertos}

\vspace{2mm}

Dado el curcuito de la siguiente figura:

\begin{center}
\includegraphics[scale=1]{prob03_wqs}
\end{center}

Determine el valor de los cuatro paramétros $h$ del circuito anterior.

\solution{
El parámetro $h_{11}$ se determina con
\begin{align*}
\mathbf{h}_{11} = \dfrac{\mathbf{V}_1}{\mathbf{I}_1}\Bigg|_{\mathbf{V}_2=0}
\end{align*}
aplicando LTK en el puerto de entrada
\begin{align*}
V_1-400I_1-1200I_1=0\\
V_1=1600I_1\\
\dfrac{V_1}{I_1}=1600
\end{align*}
por lo tanto
\begin{align*}
h_{11}=1600\,\Omega
\end{align*}
\calif{2 pts}

El parametro $h_{12}$ se detemina con
\begin{align*}
\mathbf{h}_{12} = \dfrac{\mathbf{V}_1}{\mathbf{V}_2}\Bigg|_{\mathbf{I}_1=0}
\end{align*}
se observa que cuando el puerto de entrada se abre
\begin{align*}
V_1-400I_1-1200I_1=0\\
V_1=0
\end{align*}
y como $V_1$ también es la tensión de ambas terminales del aplificador
\begin{align*}
\dfrac{0}{V_2}=0
\end{align*} 
por lo que 
\begin{align*}
h_{12}=0
\end{align*}
\calif{2 pts}

El parametro $h_{21}$ se detemina con
\begin{align*}
\mathbf{h}_{21} = \dfrac{\mathbf{I}_2}{\mathbf{I}_1}\Bigg|_{\mathbf{V}_2=0}
\end{align*}
se observa que $V_2=0$ y que la corriente de salida se puede escribir como
\begin{align*}
I_2=\dfrac{0-V_{sal}}{200}
\end{align*}
y como
\begin{align*}
\dfrac{V_{sal}500}{1000+500}=\dfrac{V_1 1200}{1200+400}\\
\dfrac{V_{sal}}{3}=\dfrac{3V_1}{4}\\
\dfrac{V_{sal}}{3}=\dfrac{3(400+1200I_1)}{4}\\
V_{sal}=3600I_1
\end{align*}
por lo que 
\begin{align*}
I_2=-\dfrac{V_{sal}}{200}=-\dfrac{3600I_1}{200}\\
\dfrac{I_2}{I_1}=-18
\end{align*}
por tanto
\begin{align*}
h_{21}=-18
\end{align*}
\calif{2 pts}

El parametro $h_{22}$ se detemina con
\begin{align*}
\mathbf{h}_{22} = \dfrac{\mathbf{I}_2}{\mathbf{V}_2}\Bigg|_{\mathbf{I}_1=0}
\end{align*}
se observa que $V_{sal}=0$ dado que las tensiones en las terminales inversora y no inversora es la misma, de manera que
\begin{align*}
\dfrac{V_2-V_{sal}}{200}=I_2\\
\dfrac{V_2}{200}=I_2\\
\dfrac{I_2}{V2}=\dfrac{1}{200}
\end{align*}
por lo que
\begin{align*}
h_{22}=\dfrac{1}{200}\,S=0.005\,S=5\,mS
\end{align*}
\calif{2 pts}

}


\problem{\clearpage}
%% 3er Examen parcial S2 2025
%% Saúl Guadamuz

\Problema{10}{Redes de dos puertos}

\vspace{2mm}

El circuito de la siguiente figura es el modelo híbrido de un amplificador tipo emisor común:

\begin{center}
\includegraphics[scale=1]{prob04}
\end{center}

Sus parámetros son $h_{11} = 45 \, \Omega$, $h_{12} = 5 \times 10^{-4}$, $h_{21} = 80$ y $h_{22} = 12.5 \, \mu S$.

\begin{subpunto}
\item Calcule $R_L$ tal que $\dfrac{\mathbf{I}_2}{\mathbf{I}_1} = 79$. \partialPoints{3}

\solution{
Tomando las ecuaciones de puerto a partir de los parámetros h y la relación eléctrica en el puerto de salida se tiene:
\begin{itemize}
\item $\mathbf{V_1} = 45\mathbf{I_1}+5\times 10^{-4}\mathbf{V_2}$
\item $\mathbf{I_2} = 80\mathbf{I_1}+12.5\mu\mathbf{V_2}$
\item $\mathbf{V_2} = -\mathbf{I_2}R_L$
\item $\dfrac{\mathbf{I_2}}{\mathbf{I_1}} = 79$
\end{itemize}

Tomando la segunda y la tercera ecuación

\begin{align*}
	\mathbf{I_2} &= 80\mathbf{I_1}+12.5\mu\left(-\mathbf{I_2}R_L\right)\\
	\mathbf{I_2}\left(1+12.5\mu R_L\right) &= 80\mathbf{I_1}\\
	\dfrac{\mathbf{I_2}}{\mathbf{I_1}}\left(1+12.5\mu R_L\right) &= 80\\
	79(1+12.5\mu R_L) &= 80\\
	R_L &= 1012,66\,\Omega
\end{align*}

}

\item Calcule la $R_{in} = \dfrac{\mathbf{V}_1}{\mathbf{I}_1}$ resultante con la $R_L$ calculada en el punto anterior. \partialPoints{2}

\solution{
Tomando la primera ecuación del punto anterior:

\begin{align*}
	\mathbf{V_1} &= 45\mathbf{I_1}+5\times 10^{-4}\mathbf{V_2}\\
	\mathbf{V_1} &= 45\mathbf{I_1}+5\times 10^{-4}(-\mathbf{I_2}R_L)\\
	\mathbf{V_1} &= 45\mathbf{I_1}+5\times 10^{-4}(-79\mathbf{I_1}1012.66)\\
	\dfrac{\mathbf{V_1}}{\mathbf{I_1}} &= 5\\
	R_{in} &= 5\,\Omega
\end{align*}
}

\item Determine el rango que puede tener $h_{22}$ para que se cumpla que $R_{in} \leq 10 \, \Omega$. \partialPoints{3}
	
\solution{
Utilizando parte del procedimiento del punto anterior

\begin{align*}
R_{in} &\leq 10\\
h_{11}-79h_{12}\dfrac{\left[\dfrac{h_{21}}{79}-1\right]}{h_{22}} &\leq 10\\
45-79\cdot 5\times10^{-4}\dfrac{\left[\dfrac{80}{79}-1\right]}{h_{22}} &\leq 10\\
-\dfrac{1}{2000}\dfrac{1}{h_{22}} &\leq 10-45\\
\dfrac{1}{2000h_{22}} &\geq 35\\
\dfrac{1}{2000\cdot 35} &\geq h_{22}\\
14.29\mu S &\geq h_{22}
\end{align*}
}

\item Calcule el valor de $R_L$ para que $R_{in} = 10 \, \Omega$. \partialPoints{2}

\solution{
Tomando la primera ecuación de puerto se tiene:	

\begin{align*}
\mathbf{V_1} &= 45\mathbf{I_1}+5\times 10^{-4}\mathbf{V_2}\\
\mathbf{V_1} &= 45\mathbf{I_1}+5\times 10^{-4}(-\mathbf{I_2}R_L)\\
\mathbf{V_1} &= 45\mathbf{I_1}+5\times 10^{-4}(-79\mathbf{I_1}R_L)\\
\dfrac{\mathbf{V_1}}{\mathbf{I_1}} &= 45 - 79\cdot5\times 10^{-4}R_L\\
R_{in} &= 45 - 79\cdot5\times 10^{-4}R_L\\
10 &= 45 - 79\cdot5\times 10^{-4}R_L\\
-35 &= - 79\cdot5\times 10^{-4}R_L\\
R_L &= 886.08\,\Omega
\end{align*}

}

\end{subpunto}







