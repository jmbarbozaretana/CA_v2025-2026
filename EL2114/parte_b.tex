\section{Aspectos operativos}

\msubsection{Metodología}
% 
El curso se imparte bajo la modalidad de clases presenciales, donde el profesor expone los temas correspondientes a los estudiantes, apoyándose en material audiovisual. El profesor tendrá la responsabilidad de aplicar las evaluaciones respectivas y entregar los resultados con la retroalimentación correspondiente. Los estudiantes deberán dar seguimiento a los temas cubiertos en clase, realizando los ejercicios que se indique y reforzando mediante horas de estudio adicionales a su aprendizaje. 

La resolución de problemas será el vehículo principal para la asimilación de los conocimientos. Los estudiantes podrán presentar sus consultas al profesor conforme avance la clase. Para profundizar sobre algún tema específico o evacuar otras dudas, el estudiante podrá asistir también a las horas establecidas de consulta por el profesor respectivo. Por otro lado, se realizarán simulaciones utilizando herramientas computacionales como apoyo para la comprensión de los temas en estudio.

El objetivo general del curso desarrolla las siguientes habilidades de acuerdo con los atributos definidos por el \atrcite[TEC]:

\buildObjTableAtr{{Aplicar los conceptos, principios y técnicas matemáticas de análisis de circuitos eléctricos en el dominio del tiempo y en el de la frecuencia ante señales de excitación alternas sinusoidales.;{CI:M,HI:I,AP:M}}}

\InsertODS{4.4,7.1,7.3,9.1,9.c}

\msubsection{Evaluación}
% 
La evaluación se realizará por medio de pruebas parciales, sin embargo, por la naturaleza del mismo, la misma representa una acumulación de conocimientos continua. La distribución de la evaluación que se utilizará durante el semestre es la siguiente:

\begin{evaluacion}[APCF]
  Primer parcial       & 33\% & Unidad 1,2 & Sábado 13/09/2025 \\
  Segundo parcial      & 34\% & Unidad 3,4,5 & Sábado 01/11/2025  \\
  Tercer parcial       & 33\% & Unidad 6,7 & Miércoles 03/12/2025 \\[1ex] 
  Examen de reposición &      &            & Miércoles 10/12/2025\\
\end{evaluacion}

El examen de reposición se aplicará el miércoles 02 de julio en el cual se evalua toda la materia del curso.\\

Al finalizar el semestre, los estudiantes con una calificación total inferior a 67,5\% pero superior o igual a 57,5\%, tienen derecho a realizar un examen de reposición, que comprenderá la materia del curso completo. Por otro lado, la reprogramación de un examen se hará exclusivamente bajo la presentación de un dictamen médico completo. 
 

Las instrucciones para las evaluaciones incluyen, aunque no se limitan, a lo siguiente:

\begin{compactitem}[nolistsep]
\item Toda prueba de evaluación de este curso es de tipo individual.
\item Se debe apagar el teléfono celular completamente.
\item No se permite el uso de ningún tipo de calculadora programable.
\item El examen debe resolverse de forma ordenada y clara.  La
  ilegibilidad o desorden del desarrollo que imposibilite su
  comprensión conducirá a una calificación de cero en la respuesta
  correspondiente, sin derecho a aclaraciones posteriores al examen. Debe presentarse en cada pregunta que  conforme una prueba de evaluación el procedimiento o argumentación que justifique su respuesta.
\item No se aceptarán reclamos de desarrollos con lápiz, borrones o 
  corrector de lapicero.
\item Los resultados deben simplificarse al máximo, y en caso
  necesario contar con unidades, respetando la notación de ingeniería.
\end{compactitem}

\nocite{Alexander2013,Dorf,Hayt,Boylestad,Nilsson,Floyd,Johnson,LTSpice,Alvarado2008,Spiegel1991}

\addtocategory{obligatoria}{Alexander2013,Dorf}
\addtocategory{complementaria}{Hayt,Boylestad,Nilsson,Floyd,Johnson,LTSpice,Alvarado2008,Spiegel1991}

\msubsection{Bibliografía} \bibbycategory    

\newcommand{\profspace}{\vspace*{5mm}}

\msubsection{Docentes}
% 
\textbf{Campus Tecnológico Local San Carlos}

%\nobreak

\begin{tabular}{lp{106mm}}
  Grupo 50 & \underline{Dr.-Ing.\ Saúl Guadamuz Brenes} \\[2mm]
           & Maestría y doctorado en ingeniería electrónica y comunicaciones,
             Politecnico de Torino, Turín, Italia. \\
           & Especialista en teoría electromagnética y comunicaciones
             inalámbricas con experiencia en proyectos de investigación y
             desarrollo en la academia e industria. \\
  Correo-e & sguadamuz@itcr.ac.cr \\
  Consulta & M 09:45 - 11:30; 14:20 - 16:00 y J 09:45-11:30 \\
  Oficina  & Edificio E, oficina 206 \\
  Teléfono & 2401-3021 \\ 
  URL 	   & \url{https://tecdigital.tec.ac.cr} \\
           & \url{https://sites.google.com/view/saulguadamuz/home}
\end{tabular}

%\profspace
%\newpage

\textbf{Campus Tecnológico Central Cartago}

%\nobreak

\begin{tabular}{lp{106mm}}
  Grupo 1 & \underline{Dra.-Ing.\ Laura Cabrera Quirós} \\[2mm]
          & Maestría en Ingeniería Electrónica con énfasis en Sistemas
            Embebidos, Tecnológico de Costa Rica.
            Doctorado con énfasis en Inteligencia Artificial y Sistemas de Monitoreo
            Ubicuo, TU Delft, Países Bajos. \\
          & Especialista en sistemas de interacción humano-computador y computación 				afectiva y ubicua. Experiencia en proyectos de investigación y desarrollo 				con la academia e industria.\\
  Correo-e & lcabrera@itcr.ac.cr   \\
  Consulta & M 13:00-16:30\\ 
  Oficina & Edificio K1, oficina 424 \\
  Teléfono & 2550-9329 \\ 
  URL & \url{https://tecdigital.tec.ac.cr}
\end{tabular}

\profspace

\begin{tabular}{lp{106mm}}
  Grupo 2 & \underline{M.\,Sc.-Ing.\ José Miguel Barboza Retana} \\[2mm]
          & Licenciatura y Maestría en Ingeniería Electrónica con énfasis
            en Sistemas
            Microelectromecánicos (MEMS), Tecnológico de Costa Rica. \\
          & Especialista en procesamiento de señales y diseño de circuitos
            integrados
            con experiencia en proyectos de investigación en áreas biomédicas.\\
  Correo-e & jmbarboza@itcr.ac.cr   \\
  Consulta & M y V 15:00:-17:00, con cita previa.\\
  Oficina & Edificio K1, oficina 321 \\
  Teléfono & 2550-2707 \\ 
  URL & \url{https://tecdigital.tec.ac.cr}
\end{tabular}

\begin{tabular}{lp{106mm}}
  Grupo 3 & \underline{Dr.-Ing.\ William F. Quirós Solano} \\[2mm]
          & Maestría en Ingeniería Electrónica con énfasis en Sistemas
            Microelectromecánicos (MEMS), Tecnológico de Costa Rica.
            Doctorado con énfasis en Microsistemas para Aplicaciones
            Biomédicas, TU Delft, Países Bajos. \\
          & Especialista en diseño, modelado, simulación y fabricación de
            Órganos-en-chip (OoCs). Experiencia en proyectos de investigación
            y desarrollo con la academia e industria.\\
  Correo-e & wquiros@itcr.ac.cr   \\
  Consulta & V 09:30-11:30  \\
  Oficina & Edificio K1, oficina 408 \\
  Teléfono & 2550-9183 \\ 
  URL & \url{https://tecdigital.tec.ac.cr}
\end{tabular}
